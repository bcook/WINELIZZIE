\documentclass[11pt,a4paper,oneside]{article}
\renewcommand{\baselinestretch}{1.2}
\usepackage{sectsty,setspace,natbib,wasysym} 
\usepackage[top=1.00in, bottom=1.0in, left=1in, right=1.25in]{geometry} 
\usepackage{graphicx}
\usepackage{latexsym,amssymb,epsf} 
\usepackage{epstopdf}
\usepackage{hyperref}

\usepackage{fancyhdr}
\pagestyle{fancy}
\fancyhead[LO]{GHD}
\fancyhead[RO]{Cook \& Wolkovich}

\begin{document}
\bibliographystyle{/Users/Lizzie/Documents/EndnoteRelated/Bibtex/styles/amnat}

\noindent {\bf Title:} Climate change decouples drought from early wingrape harvests across/in France\\
\\
\noindent {\bf Authors:} B.I. Cook\(^{1,2}\) \& E. M. Wolkovich\(^{3, 4}\)\\
\\
\noindent \emph{$^{1}$NASA Goddard Institute for Space Studies, New York, New York, United States of America; $^{2}$Ocean and Climate Physics, Lamont-Doherty Earth Observatory, Palisades, New York, United States of America;$^{3}$Arnold Arboretum, Boston, Massachusetts, United States of America; $^{4}$Organismic \& Evolutionary Biology, Cambridge, Massachusetts, United States of America} \\
\\
\begin{abstract}
Climate change has altered the timing of winegrape harvests. Across France and globally grapes mature earlier by days and weeks compared to several decades ago. Understanding the climatic drivers of these earlier harvests requires long-term records and teasing out the often intertwined drivers of fruit maturation: temperature and drought. Here we combine long-term harvest records from across France (collated by Daux et al. 2012) with reconstructions of temperature and drought to examine the drivers of early harvest over the previous centuries and more recently. We find that temperature is a dominant and consistent driver of early harvests across the time-series with warmer temperatures driving earlier harvests. Drought, however, shows a relationship that varies across the time-series. Early harvests before the mid nineteenth century were strongly correlated with droughts; in the past several decades, however, this relationship has broken down such that drought is now only weakly correlated with early harvests. Our results suggest climate change may have fundamentally altered the drivers of early winegrape harvests across France.
\end{abstract}

\newpage
\noindent {\bf Introduction:}\\

%% To do list %%
%% Make time to keep reading and refining! %%


Climate change is expected to have a major impact on crops globally in the coming years \citep{IPCC:2014uq}. As climate shifts large changes in the production of crops are expected, including changes to the timing and quality of many harvests \citep{sacks2010,wang2011,rosen2014}. Amidst these forecasts certain crops---especially those finely tuned to climate---already show a distinct signature of climate change.\\

Winegrapes (\emph{Vitis vinifera} ssp. \emph{vinifera}) are the world's most valuable horticultural crop (\url{http://faostat.fao.org}) and have shown dramatic responses to climate change over recent decades. Across Europe \citep{Jones:2000br,schultzjones,tomasi2011,odo2012} and Australia \citep{webb2012} there has been a shift to earlier harvests with some regions reporting an advance of several weeks to over a month since the 1970s \citep{Duchene:2005bd,Seguin2005} and others an advance of over two weeks per decade \citep{webb2011}. Concomitant with these changes have been shifts in wine quality ratings \citep{jones2005} and in metrics tied to wine quality \citep{mori2007}, including sugar and acid levels at grape maturity \citep{Jones:2000br}.\\

Such responses are not surprising as winegrapes are highly attuned to climate. This is especially apparent in winegrape phenology---the seasonal timing of key annual events including flowering, veraison (turning of the color and softness of berries) and harvest. Temperature and precipitation are expected to combine to influence the timing of these events, though wind, light and how all these abiotic forces are experienced by vines through the full expression of the local environment (or \emph{terroir}) also play a role \citep{Gladstones2011}. Warmer temperatures generally speed phenological events from flowering to harvest and greater precipitation generally delays events \citep{jones2013}, thus leading to the earliest harvests generally in years where growing seasons were characterized by higher temperatures and drought \citep{Jones:2000br}. Within these broad outlines both temperature and precipitation can have highly variable effects, depending on their timing and how extreme they are. For example, high temperatures can damage leaf and grape tissues \citep{greer2010,Gladstones2011} and heavy rain can burst grape clusters or promote rot \citep{jones2013}. An ideal harvest is expected from a long, warm summer---giving vines and their grapes sufficient thermal time to grow and mature---with sufficient early-season rains followed by a later season drought, giving vines resources to grow in the early season, but shifting them away from vegetative growth and towards greater investment in fruit production mid-season \citep{chaves2010,jones2013,baciocco2014}. Overall, both precipitation \citep{vanlee2009} and temperature \citep{baciocco2014} appear to control wine quality and the timing of harvest \citep{odo2012,webb2012}, though temperature is sometimes suggested to be more critical to winegrape phenology \citep{coombe1987} and many phenological analyses have focused solely on temperature \citep[e.g.,][]{jones2005}.\\
% Could use better citation for earliest harvest are hot and dry.
% Webb 2012 seems to find similar responses for temp and soil moisture, especially in consistency and sensitivity

Temperature is so critical to winegrape phenology that it has been used to back forecast climate \citep[e.g.,][]{Chuine:2004fk,Cortazar-Atauri:2010um}. Some limitations to this via varietal differences and changes in management ... \\

Yet while much work has looked at how climate impacts the timing of winegrape harvests and in turn other work has used long-term records of harvest to back forecast climate trends, there have been few efforts to examine how consistently climate affects grape harvest dates over time. In general studies examining how climate impacts harvest dates have used relatively shorter records \citep[on the scale of 30-40 years, e.g.,][]{Duchene:2005bd,tomasi2011,webb2012} and such work using longer-term records has been hampered by the paucity of long-term climate data. Here, we combine newly-available reconstructions of temperature and PDSI to examine how stationary climate effects on GHD have been over time, especially in recent time with climate change. \\
% Webb 2012 had 64 years total (most 25 years)
% Tomasi is 1964-2009 is 45 years
% Duchene 2005 is 1972 onward (32 years) and they have more phenological data that goes back further (60s I think) but no met data then.


\noindent {\bf Miscellaneous sentences:}\\
Domestication of winegrapes in Europe resulted in the diversity of grapes we have today that can be grown across diverse climates, from northern Germany to southern Italy.\\
Together with other attributes of a site (e.g., soil) climate is a major component of `terroir'--- which strongly influences wine quality.\\

\noindent Miscellaneous stuff we may need to say:\\
Other things matter also to GHD: management, varieties etc., but climate is queen.\\
Nonlinearity -- giving some examples of what happens at very high temperatures.\\

\noindent Misc (some of these points may be key for the discussion):
\\
Harvest correlates with veraison and other stages; GDD and Huglin etc. were no better than averages for predicting phenological event dates; {\bf less variability in phenological events post 1990} \citep{tomasi2011}. $\rightarrow$ Perhaps we should highlight this narrowing of variation a little then? \\

Water status affects leaf temperatures; yearly variation is going down and seems to mean more uniform quality \citep{Seguin2005}.

\bibliography{/Users/Lizzie/Documents/EndnoteRelated/Bibtex/LizzieMainMinimal}

\end{document}

