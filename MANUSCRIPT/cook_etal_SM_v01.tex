%% Template for a preprint Letter or Article for submission
%% to the journal Nature.
%% Written by Peter Czoschke, 26 February 2004
%%

\documentclass[12pt]{article}

%% make sure you have the nature.cls and naturemag.bst files where
%% LaTeX can find them

\usepackage{scicite,graphicx,rotating}
\usepackage{times}


\title{Supplementary Material: 
``Climate change decouples drought from early winegrape harvests in France''}

%% Notice placement of commas and superscripts and use of &
%% in the author list
%/Users/bcook/Desktop/WINELIZZIE/MANUSCRIPT/figures_eps_png/SUPP_fig_12_JJA_clim_regplots.png

\author{Benjamin I Cook$^{1,2}$ \& Elizabeth M Wolkovich$^{3,4}$}


\begin{document}

\maketitle

\section*{Regional Harvest Date Analyses} %EMW: This section is great!
\noindent To analyze each region individually, we regressed each harvest date time series used in GHD-Core against CRU climate data averaged within one degree of that site location. As with GHD-Core, temperature is the dominant control on harvest in all eight regional series (Supplementary Figures 4--11), and all the temperature regressions are highly significant ($p\le0.01$). For the period 1901--1980, the temperature sensitivities (regression slopes) across sites range from $-3.70$ days \textsuperscript{o}C\textsuperscript{--1} at SRv to $-7.39$ days \textsuperscript{o}C\textsuperscript{--1} at Cha1. Significant ($p\le0.05$) moisture sensitivities (precipitation and PDSI) are also found at seven of the eight sites: Als, Bor, Bur, Cha1, Lan, LLV, and Swi. At the eighth site, SRv, the precipitation regression is marginally significant ($p=0.055$) and is insignificant for PDSI.\\
\indent At all sites, temperature sensitivities are still significant for the 1981--2007 interval, indicating that temperature remains an important control on winegrape phenology in more recent decades. By contrast, most sites show a significant weakening of the two moisture sensitivities. Both the precipitation and PDSI regressions with harvest dates become (or remain) insignificant ($p>0.05$) during 1981--2007 for Bor, Bur, Cha1, Lan, LLV, and Swi. At SRv, precipitation was already only marginally significant for 1901--1980, and this regression weakens considerably. At Als, the PDSI regression becomes insignificant, while precipitation does not change appreciably. Analyses of the individual regional harvest date series therefore support the general conclusions drawn from analyses of the GHD-Core composite index, demonstrating an overall large-scale weakening of moisture controls on grape harvest dates in France and Switzerland.

\section*{Temperature versus Moisture Comparisons}
\noindent We hypothesize that, in Western Europe, moisture impacts on winegrape phenology occur primarily through the indirect effect of moisture on growing season temperatures. Further, we argue that because of anthropogenic warming trends in recent decades, winegrape phenology has become decoupled from moisture variability. Here we, present evidence in support of these hypotheses.\\ %Yeah!
\indent A strong connection between soil moisture and warm season temperatures over the Mediterranean and Western Europe has been well established \cite{Fischer2007,Miralles2014}, occurring primarily through the control of soil moisture on surface energy partitioning. In regions where evapotranspiration is limited by the supply of moisture at the surface, wetter soils will lead to increases in evapotranspiration and latent heat flux at the expense of sensible heating, keeping surface soil and air temperatures relatively cool. Conversely, if soils are dry, sensible heat fluxes will dominate, and soil and air temperatures will be higher. If these are the primary physics operating in our GHD-Core region, we would therefore expect negative relationships between temperature and precipitation or PDSI during the spring and summer.\\
\indent Indeed, for both MJJ (Supplementary Figure 12) and JJA (Supplementary Figure 14) we can see significant negative relationships between both temperature and precipitation and PDSI during the 1901--1980 interval. The relationships are generally stronger during JJA, when solar energy inputs are larger and land-atmosphere interactions are expected to be stronger. During the more recent interval (1981--2007), the temperature-moisture relationships generally weaken, especially during JJA. Additionally, temperatures during 1981--2007 are substantially warmer compared to 1901--1980, reflecting recent warming trends driven primarily by anthropogenic greenhouse gas forcing. This additional warmth (unrelated to moisture variability), combined with the apparent weakening of the temperature-moisture coupling strength itself, supports our conclusions regarding the mechanisms by which moisture influences grape harvest date variability and the recent weakening of this relationship.

\section*{Resampling Analyses}
\noindent Interpretations of the composite climate anomaly analyses (Figure 4, main manuscript) and comparisons of climate distributions (Supplemental Figure 15, top panel) for extreme early harvest years may be affected by sampling uncertainties or biases, especially during the post 1980 period when the number of years to sample from is small. For example, from 1600--1980, 72 years qualify as early harvests with GHD-Core anomalies of $-7.67$ days or earlier. Because of the terminal years in the climate reconstructions (2012 for PDSI, 2002 for temperature, 2000 for precipitation), however, from 1981 to 2007 only 18 years qualify for PDSI, 13 years for temperature, and 11 years for precipitation. To estimate the impact on our analyses of these potential sampling uncertainties, we conducted a resampling analysis where, in each of 10,000 iterations, we resampled climate anomalies from the early harvest years with replacement for the 1600--1980 period. In each iteration, we resampled $n$ times, where $n$ is equal to the number of early harvest years from 1981--2007 (so 11 for precipitation, 13 for temperature, and 18 for PDSI).\\
\indent Kernal density functions for the 10,000 resampled mean precipitation and PDSI anomalies are shown in the bottom panels of Supplemental Figure 15. As in the main analysis, PDSI shows the strongest shift with early harvest dates before and after 1980. Repeating the One Sided Student's t-test during each of these iterations found significant differences ($p\le0.05$) in 99\% of cases, indicating our earlier finding (drought, as represented by PDSI, is no longer associated with early harvests post-1980) is quite robust. The resampling results were not as robust for precipitation: only 29\% of the iterations found marginally significantly ($p\le0.10$) drier conditions during the pre-1980 period compared to post-1980. In the precipitation reconstruction, however, we are severely limited in our sample size post-1980 (only 11 years, compared to 18 for PDSI). Our conclusions, however, are still strongly supported by both the observational analysis (using 20\textsuperscript{th} century climate data) and the PDSI reconstruction.

\section*{Viticultural shifts}
% Okay, see if you hate the below. I moved things around so the go chronologically and I suggest we just cut the Languedoc findings. If they were very compelling I would keep them but the periods seem way too short to see anything so I think the discussion of the issue I have written below is stronger. If you like this we can cut all the Languedoc figures and any mention in the main text (if we have any... I don't think we do, nor should we). Also, rootstock is one word.
\subsection*{Phylloxera}
\noindent In 1860s \cite{Lachiver1988}, an exotic species of aphid (\emph{Daktulosphaira vitifoliae}, commonly known as grape phylloxera) was introduced to France. This resulted in a severe blight and destruction of many vineyards across France \cite{Lachiver1988,Loubere1978,Loubere1990,Paul1996,Bouquet2002}. Large-scale recovery of the vineyards and wine industry began in the late 1800s and early 1900s with the grafting of European vines onto phylloxera resistant rootstock from wild \emph{Vitis} species native to the United States. There was thus a substantial shift in rootstock composition beginning around the turn of the 20\textsuperscript{th} century---alongside planting of many new varieties as well as hybrid grape varieties \cite{Lachiver1988,Bouquet2002}---changes that could possibly affect our interpretation of climatic effects on harvest date.\\
\indent To investigate this, we calculated regression models between GHD-Core and JJA climate (temperature, precipitation, PDSI) from the climate reconstructions (instrumental data is not available prior to 1901) (Supplemental Figure 18) from before to after the outbreak and replanting. Across the three intervals: 1800--1850 (prior to the phylloxera epidemic), 1851--1900 (during the phylloxera epidemic), and 1901--1980 (after widespread grafting occurred) results are generally consistent. For temperature, sensitivities, range from --4.97 to --6.56 days \textsuperscript{o}C\textsuperscript{--1} with R\textsuperscript{2} values  from 0.359 to 0.464, similar to results found in the instrumental regressions for this season (Supplemental Figure 13). Similar results are found for precipitation and PDSI, although precipitation is only marginally significant for 1800--1850 (PDSI is still highly significant for this period, however). \\
Given the lack of evidence for any systematic change in sensitivities pre- and post- phylloxera, we conclude that this event is unlikely to have significantly affected our climate based analyses and interpretations. Further, as the period of 1900-1980 included substantial changes in viticultural practices (e.g., widespread mechanization, French laws mandating planting and harvest limits through appellation d'origine contr\^ol/'ee (AOC) and changes in varieties and clones planted---see below, \cite{Loubere1990,Paul1996}), the lack of substantial changes in climatic sensitivities in this period compared to periods before suggests gross changes in viticultural management do not impact phenology strongly, as least when compared to the impact of climate on phenology.
\subsection*{Clonal selection}
\noindent In the mid to late twentieth century vineyard managers began to re-plant many of their vineyards in order to produce higher quality wines through improved matching of varieties to local terroir and improved selection of clones of particular varieties \cite{Bouquet2002}. Such shifts in the composition of vineyards could introduce changes over time that would influence vine phenology and grape harvest dates. These shifts, however, occurred across different regions at different times \cite{Bouquet2002}, making it difficult for us to test for their impact on phenology through a comparison of different periods as we did for the phylloxera outbreak (see above). Yet two of other findings suggest the impact of changes in varieties and clones planted do not bias or affect our results. First, our findings are relatively robust when considered across different regions  (Supplemental Figures 4-11) that---in addition to planting substantially different varieties and clones---had different timings of major re-plantings. Second, our analysis to test for possible effects of the phylloxera epidemic and subsequent replanting (see above) found no dramatic changes in climate sensitivities of harvest dates. As the replanting after phylloxera included planting of different varieties (including hybrids) we expect that if this change had no strong impact on climate sensitivity then the comparatively smaller change of clonal selection would not impact our findings. Finally, experimental studies show the impact of climate on vine phenology is greater than variety \cite{vanleeuwen2004}.

\section*{Multiple Regression Analyses} %EMW: Edits, as promised. See what you think. I gave up trying to say why we did this analysis and just presented it in a way I thought was fair but did not make our results sound any weaker. 
\noindent Temperature, precipitation, and PDSI are all significant predictors of grape harvest dates as analyzed by our univariate approach presented in the main text. We additionally investigated if our results would change when these variables were included in a multiple regression framework using climate data from the instrumental CRU climate grids (results shown in Tables 9 for 1901--1980 and Table 10  for 1981--2007). These results highlight that temperature is the dominant driver on grape harvest date, explaining most of the variation (R\textsuperscript{2}) and being one of the top models (based on $\delta$AIC) for both time periods. Addition of precipitation and/or PDSI to a model containing temperature provides only small improvements in explanatory power, especially in 1901-1980 time period (AIC values generally increase when precipitation or PDSI are added). Additionally, precipitation and PDSI generally show a lower impact on harvest date (i.e., smaller and nonsignificant, at $p>0.05$, model coefficients) when included in a model with temperature. These results suggest that the major impact of moisture variability is not on grape phenology directly but, instead, is through the modulation of temperatures (where drier equals warmer, via the mechanisms discussed previously). These analyses further highlight the shift in how climate impacts harvest dates after 1980. Though nonsignificant, the regression coefficients for precipitation and PDSI before 1980 all indicate moisture delays harvest. After 1980, in four of the five multiple regression models these coefficients become negative. Further the sign of the interaction between temperature and moisture shifts in all models after 1980, suggesting a fundamental change in the relationship of these climate variables, as discussed above in `Temperature versus Moisture Comparisons}.' [BEN: Add a little more here? Or cut?]

\pagebreak
\bibliographystyle{naturemag}
\bibliography{/Users/bcook/Dropbox/LATEX/mylib.bib}   % name your BibTeX data base

%% TABLE: MEAN DATES
\begin{sidewaystable}
%\begin{table}
\small
\caption{\small Regional grape harvest date series from DAUX used in construction of the GHD-Core composite series. Included are the three letter codes for each site, their geographic locations (units of decimal degrees), and their mean harvest dates (day of year) for various intervals.}
\centering
\begin{tabular}{l l | r r | c c c c c p{5cm} |}
\hline
 \bf GHD Series & \bf Site Code & \bf Lat & \bf Lon & \bf 1600-1900 & \bf 1600-1980 & \bf 1901-1950 & \bf 1951-1980 & \bf 1981-2007 \\
\hline
Alsace	& Als & 48.17 & 7.28 & 282.81 & 281.87 & 272.12 & 287.16	 & 277.70 \\
Bordeaux	& Bor	& 45.18 & -0.75	& 269.01 & 	268.22	&  263.85 &  270.41 &  259.75\\
Burgundy	& Bur	& 47.32	& 5.04	& 269.92	& 270.07	& 269.44	& 272.58	& 262.15\\
Champagne 1	& Cha1	& 47.98	& 4.28	& 266.88	& 267.52	& 267.01	& 270.02	& 264.92\\
Langeudoc & Lan	& 43.60 & 3.87	& 272.43 &	270.28	& 260.97	& 266.88	& 263.40\\
Low Loire Valley	& LLV	& 47.15	& 0.22	& 286.12	& 284.61	& 282.69	& 282.56	& 275.33\\
Southern Rhone Valley	& SRv	& 43.98	& 5.05	& 269.20	& 269.10	& 268.84	& 268.46	& 257.87\\
Switzerland (Lake Geneva)	& Swi	& 46.57	& 6.52	& 286.39	& 283.87	& 273.86	& 275.22	& 263.00\\
\hline
\end{tabular}
\end{sidewaystable}

%% TABLE: SERIAL COMPLETION
% FINAL CHECK: DONE
\begin{table}
\small
\caption{\small For each regional grape harvest date series used in GHD-Core, the fraction of years with observations for various time intervals.}
\centering
\begin{tabular}{l c c c c c}
\hline
 & \bf 1354-2007 & \bf 1600-1900 & \bf 1600-2007 & \bf 1800-2007 & \bf 1900-2007\\
\hline
Als	& 0.401 & 0.578 & 0.642 & 0.837	 & 0.824\\
Bor	 & 0.500 & 0.648 & 0.738 & 0.995 & 0.991\\
Bur & 	0.925	& 0.997	& 0.990	& 0.986	& 0.972\\
Cha1	& 0.280	& 0.259	& 0.449	& 0.880	& 0.981\\
Lan	& 0.442 & 0.635	& 0.689	& 0.438	& 0.833\\
LLV	& 0.310	& 0.332	& 0.498	& 0.976	& 0.963\\
SRv	& 0.690	& 0.970	& 0.951	& 0.942	& 0.898\\
Swi 	& 0.749	& 1.00 & 1.00	& 1.00	& 1.00\\
\hline
\end{tabular}
\end{table}
%\end{sidewaystable}

%% TABLE: CROSS SITE CORRELATIONS
% FINAL CHECK: DONE
\begin{table}
\small
\caption{\small Spearman's rank correlations (1354-2007) between the various regional grape harvest date series used to construct GHD-Core.}
\centering
\begin{tabular}{c c c c c c c c c}
\hline
& Als & Bor & Bur & Cha1 & Lan & LLV & SRv & Swi \\
\hline
Als &1.00 &  &  &  &  &  &  & \\
Bor & 0.601 & 1.00 &  &  &  &  &  & \\
Bur & 0.574 & 0.615 & 1.00 &  &  &  &  & \\
Cha1 & 0.632 & 0.675 & 0.799 & 1.00 &  &  &  & \\
Lan & 0.550 & 0.467 & 0.450 & 0.655 & 1.00 &  &  & \\
LLV & 0.503 & 0.712 & 0.776 & 0.709 & 0.359 & 1.00 &  & \\
SRv & 0.414 & 0.249 & 0.456 & 0.346 & 0.765 & 0.456 & 1.00 & \\
Swi & 0.604 & 0.554 & 0.562 & 0.569 & 0.705 & 0.697 & 0.501 & 1.00\\
\hline
\end{tabular}
%\end{sidewaystable}
\end{table}

%% TABLE: GHD ANOMALIES
% FINAL CHECK: DONE
\begin{table}
\small
\caption{\small Day of year anomalies in the regional grape harvest date series used to construct GHD-Core. Anomalies are calculated relative to the baseline mean dates from 1600-1900. Also included are results for the GHD-Core index and GHD-All, composited from all 27 regional grape harvest date series in DAUX. For GHD-Core and GHD-All, the harvest date anomalies were calculated for the regional series individually before averaging together. Because the regional grape harvest date series are not all equally represented in all years, anomalies in GHD-Core and GHD-All are slightly non-zero for the baseline averaging period (1600--1900).}
\centering
\begin{tabular}{l c c c c c}
\hline
& \bf 1600-1900 & \bf 1600-1980 & \bf 1901-1950 & \bf 1951-1980 & \bf 1981-2007\\
\hline
Als	& 0	& -0.94 & -10.69 & 4.35 & -5.11\\
Bor	& 0 & -0.79 & -5.16 & 1.40 & -9.26\\
Bur	& 0	& 0.15	& -0.48	& 2.66	& -7.78\\
Cha1	& 0	& 0.64	& 0.13	& 3.14	& -1.96\\
Lan & 0 & -2.14 & -11.46 & -5.55 & -9.03\\
LLV	& 0	& -1.51	& -3.42	& -3.56	& -10.79\\
SRv & 0	& -0.10	& -0.36	& -0.74	& -11.33\\
Swi	& 0	& -2.52	& -12.53	& -11.17	& -23.39\\
\hline
GHD-Core & -0.32 & -1.02	& -5.16 & -1.11 & -10.24\\
GHD-All	& -0.25 & -0.85 & -5.13 & 0.36 & -8.91\\
\hline
\end{tabular}
%\end{sidewaystable}
\end{table}

%% TABLE: GHD STANDARD DEVIATIONS
% FINAL CHECK: DONE
\begin{table}
\small
\caption{\small As Table 4, but for inter-annual standard deviation.}
\centering
\begin{tabular}{l c c c c c}
\hline
& \bf 1600-1900 & \bf 1600-1980 & \bf 1901-1950 & \bf 1951-1980 & \bf 1981-2007\\
\hline
Als	& 8.74	& 9.56	& 9.12	& 7.09	& 8.74\\
Bor	& 9.51	& 9.00	& 6.27	& 7.00	& 9.56\\
Bur	& 9.61 & 9.19 & 6.55 & 8.11 & 7.97\\
Cha1 & 8.81 & 8.88 & 7.85 & 10.12 & 8.64\\
Lan & 8.63 & 9.03 & 6.51 & 5.69 & 5.39\\
LLV	& 10.29 & 9.21 & 6.56 & 8.01 & 7.41\\
SRv & 8.68 & 8.46 & 8.62 & 5.55 & 5.93\\
Swi	& 10.09	& 10.71	& 7.19	& 6.68	& 8.11\\
\hline
GHD-Core & 7.67 & 7.49 & 5.39 & 6.24 & 7.10\\
GHD-All	& 7.03 & 7.02 & 5.55 & 6.55 & 6.75\\
\hline
\end{tabular}
%\end{sidewaystable}
\end{table}

\begin{sidewaystable}[ht]
\centering
\caption{Coefficients and p-values from ordered logit models of wine quality data (on a scale of 0 to 5) and grape harvest dates (GHD) and CRU May-July seasonal temperatures for the periods 1900-1980 and 1981-2001. For more details on data and analyses see the Methods section in the main text.} 
\begin{tabular}{|r||l|l||l|l|}
  \hline
 & GHD: 1900-1980 & GHD: 1981-2001 & Temp: 1900-1980 & Temp: 1981-2001 \\ 
  \hline
Red Bordeaux & -0.117 ($<$0.01) & -0.133 (0.03) & 1.244 ($<$0.01) & 1.308 (0.01) \\ 
  White Bordeaux & -0.084 ($<$0.01) & -0.079 (0.14) & 0.951 ($<$0.01) & 1.109 (0.02) \\ 
  Red Burgundy & -0.102 ($<$0.01) & -0.101 (0.18) & 0.403 (0.06) & 0.612 (0.23) \\ 
  White Burgundy & -0.093 (0.02) & -0.144 (0.07) & 0.564 (0.04) & 1.262 (0.03) \\ 
   \hline
\end{tabular}
\end{sidewaystable}% latex table generated in R 3.1.2 by xtable 1.7-4 package
% Sun Apr 19 19:38:06 2015

\begin{sidewaystable}[ht]
\centering
\caption{Coefficients and p-values from  ordered logit models of wine quality data (on a scale of 0 to 5) and CRU May-July seasonal precipitation and Palmer Drought Severity Index (PDSI) for the periods 1900-1980 and 1981-2001. For more details on data and analyses see the Methods section in the main text.} 
\begin{tabular}{|r||l|l||l|l|}
  \hline
 & Prec: 1900-1980 & Prec: 1981-2001 & PDSI: 1900-1980 & PDSI: 1981-2001 \\ 
  \hline
Red Bordeaux & -0.013 ($<$0.01) & -0.011 (0.12) & -0.457 ($<$0.01) & -0.119 (0.61) \\ 
  White Bordeaux & -0.014 ($<$0.01) & -0.014 (0.06) & -0.291 (0.02) & -0.234 (0.32) \\ 
  Red Burgundy & -0.018 ($<$0.01) & 0 (0.96) & -0.273 (0.03) & 0.12 (0.64) \\ 
  White Burgundy & -0.011 (0.05) & -0.013 (0.08) & -0.101 (0.52) & -0.199 (0.41) \\ 
   \hline
\end{tabular}
\end{sidewaystable}

% Detrended vs Non-detrended statistics comparison: 1901-1980
\begin{sidewaystable}
\small
\caption{\small Comparison of regression statistics between GHD-Core and the CRU climate data (May-June-July, MJJ) for 1901-1980. Shown are results for cases where all data was linearly detrended over time prior to the regression calculation (detrended) versus the case where no detrendeding occured (normal).}
\centering
\begin{tabular}{l c c c c c c}
\hline
& \bf Slope (normal) & \bf Slope (detrended) & \bf R2 (normal) & \bf R2 (detrended) & \bf p-val (normal) & \bf p-val (detrended)\\
\hline
\bf 1901--1981 & &  &  & & & \\
Temperature & -6.044534 & -6.218814	& 0.704 & 0.780 & $<0.00000001$ & $<0.00000001$\\
Precipitation & 0.068827  & 0.068518    & 0.241 & 0.252 & 0.000004 & 0.000002\\
PDSI & 1.682021  & 1.667401   & 0.133 & 0.137 & 0.000890 & 0.000709\\
\hline
\bf 1981--2007 & &  &  & & & \\
Temperature & -6.354566 & -5.038086	& 0.642	 & 0.473 & 0.000001 & 0.000073\\
Precipitation & 0.033541 & 0.019836  & 0.051 & 0.030 & 0.258433 & 0.384498\\
PDSI & 1.067015 & 0.429598   & 0.043	 & 0.012 & 0.301859 & 0.592694\\
\hline
\end{tabular}
%\end{sidewaystable}
\end{sidewaystable}

%% TABLE: MULTIPLE REGRESSION 1901-1980
% FINAL CHECK: DONE
\begin{sidewaystable}
\small
\caption{\small Single and multivariate regressions between the CRU instrumental climate data (MJJ, 1901--1980) and GHD-Core. Predictors are in the first column (temperature, precipitation, PDSI), followed by the coefficient of determination for the full regression model (R\textsuperscript{2}), Aikaike Information Criteria (AIC), and regression coefficients and significance levels for the first (x1) and (where applicable) second (x2) and third (x3) predictors. The bottom two rows are for Temp+Precip and Temp+PDSI regression models with interactions terms}
\centering
\begin{tabular}{l c c c c c}
\hline
Predictor(s) & R\textsuperscript{2} & AIC & x1 (p) & x2 (p) & x3 (p)\\
\hline
Temp & 0.704 & 421.72 & -6.04 ($<0.001$) & NA & NA\\
Prec	 & 0.241 & 496.99 & +0.07 ($<0.001$) & NA & NA\\
PDSI & 0.133 & 507.70 & +1.68 ($<0.001$) & NA & NA\\
\hline
Temp (x1) + Prec (x2) & 0.713 & 421.27 & -5.66 ($<0.001$) & +0.02 (0.126) & NA\\
Temp (x1) + PDSI (x2) & 0.706 & 423.08 & -5.90 ($<0.001$) & +0.24 (0.436) & NA\\
Temp (x1) + Prec (x2) + PDSI (x3)	& 0.713 & 423.26 & -5.67 ($<0.001$) & +0.02 (0.190) & -0.02 (0.946)\\
Temp (x1) + Prec (x2) + [Temp X Prec] (x3)	& 0.717 & 422.2 & -3.90 ($<0.001$) & +0.16 (0.270) & -0.01 (0.317)\\
Temp (x1) + PDSI (x2) + [Temp X PDSI] (x3) & 0.708 & 424.53 & -6.14 ($<0.001$) & +4.06 (0.442) & -0.24 (0.468)\\
\hline
\end{tabular}
\end{sidewaystable}
%\end{sidewaystable}

%% TABLE: MULTIPLE REGRESSION 1981-2007
% FINAL CHECK: DONE
\begin{sidewaystable}
\small
\caption{\small As Table 9, but for the 1981--2007 period.}
\centering
\begin{tabular}{l c c c c c}
\hline
Predictor(s) & R\textsuperscript{2} & AIC & x1 (p) & x2 (p) & x3 (p)\\
\hline
Temp & 0.642 & 158.68 & -6.35 ($<0.001$) & NA & NA\\
Prec	 & 0.051 & 185.04 & +0.03 ($0.258$) & NA & NA\\
PDSI & 0.043 & 185.27 & +1.07 ($0.302$) & NA & NA\\
\hline
Temp (x1) + Prec (x2)  & 0.677 & 157.94 & -7.17 ($<0.001$) & -0.03 (0.123) & NA\\
Temp (x1) + PDSI (x2) & 0.647 & 160.3 & -6.26 ($<0.001$) & +0.37 (0.565) & NA\\
Temp (x1) + Prec (x2) + PDSI (x3)	& 0.728 & 155.27 & -7.52 ($<0.001$) & -0.06 (0.016) & +1.48 (0.048)\\
Temp (x1) + Prec (x2) + [Temp X Prec] (x3)	& 0.691 & 158.77 & -11.96 (0.022) & -0.4 (0.286) & +0.02 (0.323)\\
Temp (x1) + PDSI (x2) + [Temp X PDSI] (x3) & 0.658 & 161.52 & -6.00 ($<0.001$) & -10.34 (0.437) & +0.64 (0.420)\\
\hline
\end{tabular}
\end{sidewaystable}

\pagebreak

%% THESE COMMANDS ARE USED TO RESET THE SUPPLEMENTAL FIGURE NAMES
\renewcommand{\figurename}{Supplementary Figure}
\setcounter{figure}{0}
%\renewcommand{\thefigure}{}

\begin{figure}
\center
\includegraphics[width=1.0\columnwidth,scale=2]{SUPP_fig_01_map_sites.png}
\caption{Geographic locations for all 27 regional grape harvest date time series in DAUX. Highlighted in red are the sites that comprise the GHD-Core index: Alsace (Als), Bordeaux (Bor), Burgundy (Bur), Champagne 1 (Cha1), Languedoc (Lan), the Lower Loire Valley (LLV), the Southern Rhone Valley (SRv), and Switzerland at Lake Geneva (Swi). The dashed black box indicates the GHD-Core region (2\textsuperscript{o}W--8\textsuperscript{o}E, 43\textsuperscript{o}N--51\textsuperscript{o}N) over which climate anomalies from the CRU instrumental climate datasets and the three climate reconstructions were averaged for the various regression analyses.}
\end{figure}

\clearpage

\begin{figure}
\center
\includegraphics[width=1.0\columnwidth,scale=2]{SUPP_fig_02_numobs.png}
\caption{Number of observations (i.e., regional grape harvest date series) represented in each year of the GHD-Core index.}
\end{figure}

\begin{sidewaysfigure}
\center
\includegraphics[width=1.0\columnwidth,scale=2]{SUPP_fig_03_core_vs_sites.png}
\caption{Time series (1600--2007) of GHD-Core (blue sold line) and each individual regional grape harvest date  series (green dashed lines) used in the construction of GHD-Core.}
\end{sidewaysfigure}

\begin{figure}
\center
\includegraphics[width=.9\columnwidth,scale=2]{SUPP_fig_04_Als_MJJ_climate_onedeg_withtrend.png}
\caption{Comparisons between the grape harvest date series from ALS and May-June-July temperature, precipitation, and PDSI from the CRU 3.21 climate grids. Top panels: point-by-point Spearman's rank correlations for 1901--1980 and 1981--2007 (location of ALS is shown by the black dot). Bottom panels: linear regression plots for the same intervals against CRU climate data averaged within one degree of the site location.}
\end{figure}

\begin{figure}
\center
\includegraphics[width=.9\columnwidth,scale=2]{SUPP_fig_05_Bor_MJJ_climate_onedeg_withtrend.png}
\caption{Same as Figure 4, but for Bor.}
\end{figure}

\begin{figure}
\center
\includegraphics[width=.9\columnwidth,scale=2]{SUPP_fig_06_Bur_MJJ_climate_onedeg_withtrend.png}
\caption{Same as Figure 4, but for Bur.}
\end{figure}

\begin{figure}
\center
\includegraphics[width=.9\columnwidth,scale=2]{SUPP_fig_07_Cha1_MJJ_climate_onedeg_withtrend.png}
\caption{Same as Figure 4, but for Cha1.}
\end{figure}

\begin{figure}
\center
\includegraphics[width=.9\columnwidth,scale=2]{SUPP_fig_08_Lan_MJJ_climate_onedeg_withtrend.png}
\caption{Same as Figure 4, but for Lan.}
\end{figure}

\begin{figure}
\center
\includegraphics[width=.9\columnwidth,scale=2]{SUPP_fig_09_LLV_MJJ_climate_onedeg_withtrend.png}
\caption{Same as Figure 4, but for LLV.}
\end{figure}

\begin{figure}
\center
\includegraphics[width=.9\columnwidth,scale=2]{SUPP_fig_10_SRv_MJJ_climate_onedeg_withtrend.png}
\caption{Same as Figure 4, but for SRv.}
\end{figure}

\begin{figure}
\center
\includegraphics[width=.9\columnwidth,scale=2]{SUPP_fig_11_SWi_MJJ_climate_onedeg_withtrend.png}
\caption{Same as Figure 4, but for SWi.}
\end{figure}

\begin{figure}
\center
\includegraphics[width=1.0\columnwidth,scale=2]{SUPP_fig_12_temp_vs_moist_MJJ.png}
\caption{Temperature versus moisture (precipitation and PDSI) during May-June-July for the GHD-Core region from the CRU 3.21 climate grids for two periods: 1901--1980 and 1981--2007. Temperature during MJJ has a significant negative relationship with both precipitation and PDSI, indicating the tendency for warmer conditions during drier years. Over the more recent period, the relationship between temperature and PDSI breaks down, but the relationship with precipitation remains largely consistent.}
\end{figure}

\begin{figure}
\center
\includegraphics[width=1.0\columnwidth,scale=2]{SUPP_fig_13_JJA_clim_regplots.png}
\caption{Same as Figure 3 from the main manuscript, but for June-July-August (JJA) climate.}
\end{figure}

\begin{figure}
\center
\includegraphics[width=1.0\columnwidth,scale=2]{SUPP_fig_14_temp_vs_moist_JJA.png}
\caption{Same as Supplemental Figure 12, but for June-July-August (JJA) climate. During JJA, the temperature moisture relationship is stronger over this region and, in both precipitation and PDSI, these relationships breakdown in the recent decades (1981--2007).}
\end{figure}

\begin{figure}
\center
\includegraphics[width=1.0\columnwidth,scale=2]{SUPP_fig_15_JJA_boxplot_monte.png}
\caption{Top row: box plots of pooled precipitation and PDSI anomalies from the climate reconstructions for years with early harvest date anomalies in GHD-Core. Bottom row: kernel density functions of recalculated mean precipitation and PDSI anomalies associated with early harvest dates, based on 10,000 Monte-Carlo resamplings with replacement. Dashed vertical lines represent the mean anomalies calculated from the actual early harvest years from 1600--1980 (blue) and 1981--2007 (green).}
\end{figure}

%\begin{figure} %EMW: Cut?
%\center
%\includegraphics[width=1.0\columnwidth,scale=2]{SUPP_fig_16_clonetest_Lan.png}
%\caption{Regressions between harvest date and the instrumental data (temperature, precipitation, PDSI) for the Languedoc regional series. Interval correspond roughly to a time period when significant clonal selection at Languedoc likely occurred.}
%\end{figure}

\begin{figure}
\center
\includegraphics[width=1.0\columnwidth,scale=2]{SUPP_fig_17_clonetest_GHDcore.png}
\caption{As Supplementary Figure 16, but for GHD-Core.}
\end{figure}

\begin{figure}
\center
\includegraphics[width=1.0\columnwidth,scale=2]{SUPP_fig_18_regress_ghdcore_phylloxera.png}
\caption{Regressions between harvest date and the climate reconstructions data (temperature, precipitation, PDSI) for the GHD-Core composite series. Interval correspond to intervals prior to the introduction of phylloxera (1800--1850), during the infestation (1851--1900), and following the widespread grafting of French vines onto North American rootstock (1901--1980).}
\end{figure}

\end{document}









