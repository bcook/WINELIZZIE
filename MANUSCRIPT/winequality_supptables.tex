% Straight up stealing preamble from Eli Holmes 
%%%%%%%%%%%%%%%%%%%%%%%%%%%%%%%%%%%%%%START PREAMBLE THAT IS THE SAME FOR ALL EXAMPLES
\documentclass{article}

%Required: You must have these
\usepackage{Sweave}
\usepackage{graphicx}
\usepackage{natbib}

%Strongly recommended
  %put your figures in one place
%you'll want these for pretty captioning
\usepackage[small]{caption}
\setkeys{Gin}{width=0.8\textwidth}  %make the figs 50 perc textwidth
\setlength{\captionmargin}{30pt}
\setlength{\abovecaptionskip}{0pt}
\setlength{\belowcaptionskip}{10pt}
% manual for caption  http://www.dd.chalmers.se/latex/Docs/PDF/caption.pdf

%Optional: I like to muck with my margins and spacing in ways that LaTeX frowns on
%Here's how to do that
 \topmargin -1.5cm        
 \oddsidemargin -0.04cm   
 \evensidemargin -0.04cm  % same as oddsidemargin but for left-hand pages
 \textwidth 16.59cm
 \textheight 21.94cm 
 %\pagestyle{empty}       % Uncomment if don't want page numbers
 \parskip 7.2pt           % sets spacing between paragraphs
 %\renewcommand{\baselinestretch}{1.5} 	% Uncomment for 1.5 spacing between lines
\parindent 15pt		  % sets leading space for paragraphs
\usepackage{setspace}
%\doublespacing

\usepackage{fancyhdr}
\pagestyle{fancy}
\fancyhead[LO]{June 2015}
\fancyhead[RO]{Wine quality}
 
%%%%%%%%%%%%%%% END PREAMBLE THAT IS THE SAME FOR ALL EXAMPLES %%%%%%%%%%%%%%%%%%%%%%%

%Start of the document
\begin{document}
% \bibliographystyle{/Users/Lizzie/Documents/EndnoteRelated/Bibtex/styles/pnas}

\title{Supplemental tables for Cook \& Wolkovich: \\ Analyses of wine quality over the last 100 years}
% \author{Cook \& Wolkovich}
\date{\today}
\maketitle  %put the fancy title on
%%%%%%%%%%%%%%%%%%%%%%%%%%%%%%%%%%%%%%%%%%%%%%%%%%%
\renewcommand{\thetable}{S\arabic{table}}
\renewcommand{\thefigure}{S\arabic{figure}}

% \renewcommand{\tablename}{Table S}
% \renewcommand{\figurename}{Figure S}

%%%%%%%%%%%%%%%% Now Lizzie attempts to write code %%%%%%


% latex table generated in R 3.1.2 by xtable 1.7-4 package
% Tue Jun 16 07:39:34 2015
\begin{table}[ht]
\centering
\caption{No real caption as table is too long to fit in manuscript. I think we should include this information as two tables, see Tables S2 and D3 below.} 
\begin{tabular}{r|ll|ll|ll|ll}
  \hline
 & GHD: pre & GHD: post & Temp: pre-80 & Temp: post-80 & Prec: pre-80 & Prec: post-80 & PDSI: pre-80 & PDSI: post-80 \\ 
  \hline
Red Bordeaux & -0.117 ($<$0.01) & -0.133 (0.03) & 1.244 ($<$0.01) & 1.308 (0.01) & -0.013 ($<$0.01) & -0.011 (0.12) & -0.457 ($<$0.01) & -0.119 (0.61) \\ 
  White Bordeaux & -0.084 ($<$0.01) & -0.079 (0.14) & 0.951 ($<$0.01) & 1.109 (0.02) & -0.014 ($<$0.01) & -0.014 (0.06) & -0.291 (0.02) & -0.234 (0.32) \\ 
  Red Burgundy & -0.102 ($<$0.01) & -0.101 (0.18) & 0.403 (0.06) & 0.612 (0.23) & -0.018 ($<$0.01) & 0 (0.96) & -0.273 (0.03) & 0.12 (0.64) \\ 
  White Burgundy & -0.093 (0.02) & -0.144 (0.07) & 0.564 (0.04) & 1.262 (0.03) & -0.011 (0.05) & -0.013 (0.08) & -0.101 (0.52) & -0.199 (0.41) \\ 
   \hline
\end{tabular}
\end{table}% latex table generated in R 3.1.2 by xtable 1.7-4 package
% Tue Jun 16 07:39:34 2015
\begin{table}[ht]
\centering
\caption{Coefficients and p-values from ordered logit models of wine quality data (on a scale of 0 to 5) and grape harvest dates (GHD) and Luterbacher May-July seasonal temperatures for the periods 1900-1980 and 1981-2001. For more details on data and analyses see XXXX (Methods in main text?).} 
\begin{tabular}{||r||l|l||l|l||}
  \hline
 & GHD: 1900-1980 & GHD: 1981-2001 & Temp: 1900-1980 & Temp: 1981-2001 \\ 
  \hline
Red Bordeaux & -0.117 ($<$0.01) & -0.133 (0.03) & 1.244 ($<$0.01) & 1.308 (0.01) \\ 
  White Bordeaux & -0.084 ($<$0.01) & -0.079 (0.14) & 0.951 ($<$0.01) & 1.109 (0.02) \\ 
  Red Burgundy & -0.102 ($<$0.01) & -0.101 (0.18) & 0.403 (0.06) & 0.612 (0.23) \\ 
  White Burgundy & -0.093 (0.02) & -0.144 (0.07) & 0.564 (0.04) & 1.262 (0.03) \\ 
   \hline
\end{tabular}
\end{table}% latex table generated in R 3.1.2 by xtable 1.7-4 package
% Tue Jun 16 07:39:34 2015
\begin{table}[ht]
\centering
\caption{Coefficients and p-values from  ordered logit models of wine quality data (on a scale of 0 to 5) and Pauling May-July seasonal precipitation and Palmer Drought Severity Index (PDSI) for the periods 1900-1980 and 1981-2001. For more details on data and analyses see XXXX (Methods in main text?).} 
\begin{tabular}{||r||l|l||l|l||}
  \hline
 & Prec: 1900-1980 & Prec: 1981-2001 & PDSI: 1900-1980 & PDSI: 1981-2001 \\ 
  \hline
Red Bordeaux & -0.013 ($<$0.01) & -0.011 (0.12) & -0.457 ($<$0.01) & -0.119 (0.61) \\ 
  White Bordeaux & -0.014 ($<$0.01) & -0.014 (0.06) & -0.291 (0.02) & -0.234 (0.32) \\ 
  Red Burgundy & -0.018 ($<$0.01) & 0 (0.96) & -0.273 (0.03) & 0.12 (0.64) \\ 
  White Burgundy & -0.011 (0.05) & -0.013 (0.08) & -0.101 (0.52) & -0.199 (0.41) \\ 
   \hline
\end{tabular}
\end{table}
\end{document}
