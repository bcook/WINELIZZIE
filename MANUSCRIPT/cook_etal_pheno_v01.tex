%% Template for a preprint Letter or Article for submission
%% to the journal Nature.
%% Written by Peter Czoschke, 26 February 2004
%%

\documentclass[final]{nature}
\usepackage{url,hyperref}
\usepackage{graphicx}


%% make sure you have the nature.cls and naturemag.bst files where
%% LaTeX can find them

\bibliographystyle{naturemag}

\title{Climate change decouples drought from early winegrape harvests in France} % Not Swance?

%% Notice placement of commas and superscripts and use of &
%% in the author list

\author{Benjamin I Cook$^{1,2}$ \& Elizabeth M Wolkovich$^{3,4}$}


\begin{document}

\maketitle

\begin{affiliations}
 \item NASA Goddard Institute for Space Studies, New York City, NY, USA
 \item Ocean and Climate Physics, Lamont-Doherty Earth Observatory, Palisades, NY, USA
 \item Arnold Arboretum, Boston, MA, USA
 \item Organismic and Evolutionary Biology, Harvard University, Cambridge, MA, USA
\end{affiliations}

\begin{abstract}
Across much of the globe winegrape phenology has significantly advanced in recent decades\cite{Duchene:2005bd,Seguin2005,webb2011}. Such trends appear in step with climate change induced trends in temperature and drought---the main drivers of fruit maturation. Fully understanding how climate change contributes to these trends, however, requires analyzing winegrape phenology and its relationship to climate over a longer term context, including data predating anthropogenic interference in the climate system. Here, we investigate the climatic controls of early grape harvest dates (GHD) from 1600--2007 in France (9 sites) and Switzerland (1 site), using historical GHD data\cite{Daux2012} and independent reconstructions of temperature\cite{Luterbacher2004} and drought\cite{CookOWDA2015,Pauling2006}. Warm temperatures and drought during the late spring and early summer (May-June-July) are the primary drivers of early harvests, but in recent decades (1981--2007) drought has become decoupled from harvest timing. Historically, high summer temperatures in Western Europe, which would hasten fruit maturation, required drought conditions to generate extreme heat. The relationship between drought and temperature in this region, however, has weakened in recent decades and, with enhanced warming from anthropogenic greenhouse gases, sufficiently high temperatures for early harvests can now occur regularly without drought. Our results suggest that anthropogenic climate change may have fundamentally altered the climatic drivers of early winegrape harvests in France, with possible ramifications for viticulture management and wine quality. 
\end{abstract}

\noindent Winegrapes (\emph{Vitis vinifera} ssp. \emph{vinifera}) are the world's most valuable horticultural crop, and there is increasing evidence that warming trends have advanced winegrape harvest dates in recent decades\cite{Duchene:2005bd,Jones:2000br,schultzjones,Seguin2005,tomasi2011,odo2012,webb2012}. Harvest dates are closely connected to the timing of grape maturation which is highly sensitive to climate during the growing season. Specifically, warmer temperatures accelerate grape vine phenological development from flowering to fruit maturation and harvest, while increased precipitation tends to delay winegrape phenology\cite{jones2013}. The earliest harvests thus generally occur in years where the growing season experiences warmer temperatures and drought\cite{Jones:2000br}.\\
\indent Along with trends in harvest dates, there have also been apparent shifts in wine ratings\cite{jones2005} and other metrics of wine quality\cite{Jones:2000br,mori2007}. High quality wines are typically associated with early harvest dates in many regions\cite{Jones:2000br,jones2005}, and are also favored by warm summers with above average early-season rainfall and late season drought. This ensures the vines and grapes have sufficient heat and moisture to grow and mature early on, with dry conditions later in the year shifting them away from vegetative growth and towards greater investment in fruit production mid-season\cite{chaves2010,jones2013,baciocco2014}. Overall, both precipitation\cite{vanlee2009} and temperature\cite{baciocco2014} contribute to wine quality and the timing of harvest\cite{odo2012,webb2012}, though temperature is suggested to be most critical to winegrape phenology\cite{coombe1987,jones2005}.\\
% BIC 29-Jun-2015: Only one site in Switzerland-confirmed (7 sites in France). I added your suggested wine quality sentence to the end. I made a couple very minor edits.
\indent These shifting trends in viticulture have led to much recent research to better understand climate controls on winegrape phenology\cite{odo2012,webb2012}, especially grape harvest dates, (GHD) and quality\cite{coombe1987,jones2005,vanlee2009}.  Most research has, however, focused on relatively short, recent timescales (e.g., the last 30--40 years\cite{Duchene:2005bd,tomasi2011,webb2012}). There has thus been little consideration of the 1) longer term historical context of recent GHD trends and 2) possible non-stationarities in the relationship between winegrape phenology and climate. We address these issues by conducting a new analysis using over 400 years (1600--2007) of GHD data from Western Europe\cite{Daux2012}. From this database, we construct a multi-site GHD index (hereafter, GHD-Core) by averaging harvest date anomalies from 7 regional GHD time series across France and one site in Switzerland (see Methods for more details). We then analyze the variability and trends in GHD-Core, and compare against instrumental climate data over the 20\textsuperscript{th} century\cite{Harris2014} and proxy-based reconstructions of temperature\cite{Luterbacher2004}, precipitation\cite{Pauling2006}, and soil moisture (Palmer Drought Severity Index; PDSI)\cite{CookOWDA2015} (PDSI) back to 1600. We also test for associated shifts in wine quality for two sites (Bordeaux and Burgundy), using 100 years of wine quality ratings\cite{Broadbent2002}.\\
\indent The GHD-Core series has pronounced variability from year to year and a strong trend towards earlier dates in the latter part of 20\textsuperscript{th} century (Figure 1). The latest date in the record (Figure 1, left panel) is 1816, the so called `Year without a Summer' following the eruption of Mount Tambora in Indonesia\cite{Oppenheimer2003}. The eruption caused pronounced cooling over continental Europe during the growing season, with harvest dates in GHD-Core delayed over three weeks ($+24.8$ days). The earliest date in the record is 2003 ($-31.4$ days), coinciding with one of the worst summer heat waves in recent history\cite{Rebetz2006}. Mean harvest dates were modestly early during the first half of the 20\textsuperscript{th} century (1901--1950, $-5.2$ days), roughly average from 1951--1980 ($-1.1$ days), and substantially earlier during the most recent decades (1981--2007, $-10.24$ days) (Supplemental Table 4). The 1981--2007 mean date exceeds one full standard deviation of GHD variability calculated from the baseline averaging period (1600--1900, $\pm7.67$ days) and is significantly earlier than the full previous interval (1600--1980; One Sided Student's t-test, $p\le0.0001$). The 1981-2007 period is also earlier than the earliest previous 27 year interval (1635--1661, $-7.42$ days), although results are only marginally significant (One Sided Student's t-test, $p=0.075$).\\
\indent There are strong and significant correlations between GHD-Core and the instrumental climate data, although the strength of the moisture relationships (precipitation and PDSI) declines in recent years (Figure 2) (for individual regional GHD series, see Supplemental Figures 4--11). GHD-Core correlates negatively (Spearman's rank) with May-June-July (MJJ) temperatures across Western Europe (Figure 2, top row), indicating a strong tendency for earlier harvests during warmer conditions in late spring and early summer. Regional average (dashed box in Figure 2; 2\textsuperscript{o}W--8\textsuperscript{o}E, 43\textsuperscript{o}N--51\textsuperscript{o}N) MJJ temperatures are the single best predictor of GHD-Core (Figure 3), explaining 70\% of the variance for 1901--1980 and only weakening slightly in the more recent period ($R^2=0.64$). Notably, the slope of the regression is similar before and after 1980 (harvest dates advancing approximately $-6$ days per degree of warming), suggesting that the temperature sensitivity of harvest dates is relatively stationary over time.\\
\indent Correlations are positive, though weaker, with MJJ precipitation (Figure 2, middle row) and PDSI (Figure 2, bottom row), indicating earlier harvests during drought conditions. This may be due to direct drought impacts on fruit maturation by increasing abscisic acid production\cite{webb2012} or indirectly through feedbacks between soil moisture and air temperature. Dry soils favor sensible over latent (i.e., evapotranspiration) heating, increasing soil and air temperatures and speeding up fruit maturation. Western Europe is a region where this soil moisture-temperature interaction is thought to be especially strong\cite{Seneviratne2006} (Supplemental Figure 12, top row). These moisture-GHD relationships persist through the middle of the century (1951--1980), but become insignificant in recent decades (1981-2007) (Figure 3).\\
% BIC 29-Jun-2015: In the last sentence I meant that the moisture-temperature regressions are stronger for JJA than for MJJ. I can see why you were confused, so I changed the sentence to hopefully make this clearer.
\indent To further investigate this apparent weakening of the GHD-drought relationship, we composited climate anomalies back to 1600 during early harvest years, defined as years when GHD-Core was $-7.67$ days early or earlier (one standard deviation). For this, we used June-July-August (JJA) average climate, the closest match available to the MJJ season in the seasonally resolved climate reconstructions. In the instrumental data, the relationships between GHD-Core and temperature and precipitation weaken during JJA compared to MJJ, while PDSI improves slightly (Supplemental Figure 13). All regressions prior to 1980 are still significant, however, and JJA comparisons between GHD and moisture (precipitation and PDSI) show a similar weakening and loss of significance from 1981--2007. The temperature-moisture coupling relationships for 1901--1980 are stronger during JJA than MJJ, and both the precipitation and PDSI regressions with temperature becoming insignificant afterward (Supplemental Figure 14).\\ 
\indent Compositing the early harvest dates in GHD-Core yields 72 years from 1600--1980; from 1981--2007, the composite ranged from 11--18 years, depending on the end date of the different climate reconstructions (Figure 4). As expected, early harvests are associated with warmer than average conditions in both intervals, increasing in intensity in the more recent period (consistent with large-scale greenhouse gas forced warming trends over Europe). Composite precipitation and PDSI are dry during 1600--1980, with regional average precipitation $-11\%$ below normal and mean PDSI$=-1.1$ (indicative of a modest drought).\\
\indent After 1980, the association between dry anomalies and early harvests effectively disappears, with regional average mean precipitation only slightly below normal ($-1.3\%$) and PDSI actually wetter than average ($+0.86$). Differences in the early harvest PDSI composite pre- and post-1980 are highly significant (One Sided Student's t-test, $p\le0.001$), while only marginally significant for precipitation (One Sided Student's t-test, $p=0.08$). However, a one sample Student's t-test comparing the precipitation anomalies against a mean of zero found that only the precipitation anomalies in the pre-1980 period are significantly below normal. These results were confirmed by a Monte-Carlo analysis to test for sampling uncertainties in the composite averaging (Supplemental Figure 15). These results further support our conclusion from the 20\textsuperscript{th} century climate analyses, indicating that drought has become decoupled in recent decades as a significant driver of early harvest dates.\\
\indent Two factors have likely contributed to the diminishing importance of moisture for winegrape phenology. The first is the apparent weakening of the soil moisture-temperature relationship over Western Europe in recent decades, which is especially apparent for JJA (Supplemental Figure 14). Prior to 1981, moisture variability (as represented by precipitation and PDSI) accounts for approximately 25\% of the year to year temperature variability in this region. In more recent decades, however, these moisture-temperature regressions become insignificant. Second, with the strengthening of anthropogenic greenhouse gas induced warming, this added heating has made it easier for summers to reach critical heat thresholds needed for early harvest dates. Previously, drought conditions would have been a necessary pre-condition to reach such extremes.\\
% BIC 29-Jun-2015: I like your split of the topic sentence-just made a very minor change.
\indent Climate and harvest timing are both thought to affect wine quality, but have generally been assumed to be stationary. But if the climatic constraints on winegrape phenology are changing, then environmental effects on quality may also be non-stationary. Using wine ratings for the Bordeaux and Burgundy regions\cite{Broadbent2002}, we analyzed harvest timing and climate effects on wine quality pre- and post-1980. In these regions the likelihood of higher quality wines increases with earlier harvests and higher temperatures (see Supplemental Table 6), and these harvest date and temperature effects are generally significant and of similar magnitude before and after 1980. Higher quality wines are also favored by dry conditions pre-1980 (Supplemental Table 7), but the relationship between PDSI and quality weakens considerably after 1980 (either becoming insignificant or seeing much reduced magnitudes in the ordinal coefficients). Thus, there has been a recent decoupling between wine quality and drought, similar to the results from our climate and GHD analysis.\\
% BIC 29-Jun-2015: Looks good!
\indent Our findings---suggesting a largescale shift in how climate drives early harvests across France and Switzerland---are generally consistent across regions (Supplemental Figures 4--11).  This consistency is important for two major reasons. Firstly, winegrape varieties span a great degree of phenological diversity, and there may be related differences in their sensitivities to climate within and across regions\cite{Parker2013}. Second, both the trends in harvest dates and changes in the climate constraints could be explained by viticultural management changes in recent decades, rather than shifts in environmental forcing. We find however, good cross-site correlations across the regional series used to create GHD-Core (Supplemental Table 3; Supplemental Figure 3) and diverse regions---for example, Alsace, Champange, Burgundy and Languedoc---show findings similar to our overall results (Supplemental Figures 4--11, one notable exception was Bordeaux where climate relationships have been relatively stable over time). These regions span greatly differing varieties and management regimes that have generally not shifted similarly. This indicates some commonality to the climate signal across the regions, making it unlikely our results and interpretations are biased by one (or a few) of the GHD series. Further, irrigation, the management activity that would be most likely to complicate our climate interpretations, is generally not allowed in France, making it highly unlikely that this could explain the reduction in moisture signal in recent years.\\
% BIC 29-Jun-2015: Looks good! Added a citation for all the factors going into wine quality.
\indent Our results indicate a fundamental shift in the role of drought and moisture availability as large-scale drivers of harvest timing and wine quality across France and Switzerland. Long-term GHD records and wine quality estimates demonstrate that warm temperatures have been a consistent driver of early harvests and higher quality wines. Relationships with drought, however, have largely disappeared in recent decades, a consequence of large-scale shifts in the climate system that have decoupled high growing season temperatures from dry summers. Our results do not necessarily presage an inevitable future where wine quality is dominated by environmental changes. In reality, GHD and wine quality depend on a number of factors beyond climate---including winegrape varieties, soils, vineyard management, and winemaker practices\cite{Jackson1993,Leeuwen2013}. Our results do suggest, however, that the large-scale climatic drivers within which these generally local factors act has fundamentally shifted. Such information may be critical to wine production as climate change intensifies over the coming decades in France, Switzerland, and other wine growing regions.

%\emph{Caveats we may want to add:} Quality results are variable across regions: temperature is a stronger predictor of quality in Bordeaux compared to Burgundy (that is, quality of Bordeaux wines increases about about twice as quickly with higher temperatures compared to wines from Burgundy, except for the weird strengthening of the relationship for whites in Burgundy after 1980). Variable responses are due to probably to a bunch of things including (a) microclimatic variation, which occurs at as small as scale as the sub-vineyard and which our analyses brush over, (b) variation in the y axis of our climate variables --- we should be sure to hit this one home as higher temperature may be good now, but -- if we bring the reader back to the introduction -- at their extremes they lead to vintage failures.
%Then the body of the main text appears after the intro paragraph.
%Figure environments can be left in place in the document.
%\verb|\includegraphics| commands are ignored since Nature wants
%the figures sent as separate files and the captions are
%automatically moved to the end of the document (they are printed
%out with the \verb|\end{document}| command. However, tables must
%be manually moved to the end of the document, after the addendum.

\begin{figure}
\caption{Time series of Grape Harvest Date (GHD) anomalies from GHD-Core (left panel), composited from the Als, Bor, Bur, Cha1, Lan, LLV, SRv, and SWi regional GHD time series in the DAUX dataset. All anomalies are in units of day of year anomalies, calculated relative to the average date from 1600--1900. In the right panel, we compare normalized histograms of GHD anomalies from GHD-Core for two periods: 1600--1980 and 1981--2007.}
\end{figure}

\begin{figure}
\caption{Twentieth century analysis between climate obsevations and GHD-Core. Panels show point-by-point correlations (Spearman's rank) between GHD-Core and May-June-July temperature, precipitation, and Palmer Drought Severity Index (PDSI) for three periods: 1901--1950, 1951--1980, and 1981--2007. All the climate data are from the CRU 3.21 climate grids, described in the Methods section.}
\end{figure}

\begin{figure}
\caption{Twentieth century analysis between climate observations and GHD-Core. Panels show linear regressions between GHD-Core and May-June-July climate variables from CRU 3.21, averaged over the main GHD-Core region (2\textsuperscript{o}W--8\textsuperscript{o}E, 43\textsuperscript{o}N--51\textsuperscript{o}N). The top row shows results from 1901--1980; the bottom row for 1981--2007. Calculating the regression statistics on the detrended data yielded nearly identical results, summarized in Supplemental Table 8.}
\end{figure}

\begin{figure}
\caption{Analysis between paleo-climate reconstructions and GHD-Core. Composite average temperature, precipitation, and PDSI anomalies from the various climate reconstructions (see Methods) from years with early harvest dates ($\le-7.67$ days early). Numbers in the lower left corners indicate the number of years used to construct each composite.}
\end{figure}

%In addition, a cover letter needs to be written with the
%following:
%\begin{enumerate}
% \item A 100 word or less summary indicating on scientific grounds
%why the paper should be considered for a wide-ranging journal like
%\textsl{Nature} instead of a more narrowly focussed journal.
% \item A 100 word or less summary aimed at a non-scientific audience,
%written at the level of a national newspaper.  It may be used for
%\textsl{Nature}'s press release or other general publicity.
% \item The cover letter should state clearly what is included as the
%submission, including number of figures, supporting manuscripts
%and any Supplementary Information (specifying number of items and
%format).
 %\item The cover letter should also state the number of
%words of text in the paper; the number of figures and parts of
%figures (for example, 4 figures, comprising 16 separate panels in
%total); a rough estimate of the desired final size of figures in
%terms of number of pages; and a full current postal address,
%telephone and fax numbers, and current e-mail address.
%\end{enumerate}

%See \textsl{Nature}'s website
%(\texttt{http://www.nature.com/nature/submit/gta/index.html}) for
%complete submission guidelines.
\pagebreak
\begin{methods}
\subsection{Grape Harvest Data.}
\noindent We analyzed GHD data in the database of regional winegrape harvest time series from Western Europe compiled by Daux et al 2012 (hereafter, DAUX\cite{Daux2012}). DAUX included 27 regional composite time series of winegrape harvest dates, compiled from local vineyard and winery records going back as far as 1354. Most of these series were from France, but also included were data from Switzerland, Spain, Luxembourg, and Germany (Supplemental Figure 1). These data were ideal for climate change research applications because management practices have changed generally little over time and irrigation as a viticulture tool (which could have complicated the interpretation of climate relationships) was (and still is) largely absent, especially in France. Indeed, these data have been used previously to develop proxy-based temperature reconstructions for the region\cite{Daux2012}.\\
\indent We created a composite average index from several regional series (GHD-Core) as the focus for our analysis. Using a multi-site composite series had two main advantages. First, every regional GHD series had at least some missing values. By averaging multiple sites into a single composite index, we were able to ensure a serially complete time series back to 1600. Second, because viticulture management varies across winegrape varietals and regions, use of a composite average series should minimize the influence of local management effects (which are unlikely to be synchronous across space) and instead emphasize larger scale signals related to climate variability and change (the primary focus of our study).\\
\indent From the 27 regional GHD series available, we chose 8 sites (Supplemental Table 1) to construct GHD-Core: Alsace (Als), Bordeaux (Bor), Burgundy (Bur), Champagne 1 (Cha1), Languedoc (Lan), the Lower Loire Valley (LLV), the Southern Rhone Valley (SRv), and Switzerland at Leman Lake (SWi). All 7 regional series were over $80\%$ serially complete back to 1800 and all but one (Cha1) were over $60\%$ complete back to 1600 (Supplemental Table 2). Importantly, all 8 sites had good coverage for the most recent period (1981--2007) when we conclude that drought controls on harvest date have significantly weakened. After 1600, most years have at least 3-4 of these regional series represented; sample depth declines sharply prior to this date (Supplemental Figure 2). All analyses are thus restricted to the period from 1600--2007 (also the time period indicated by Daux et al 2012 as the most reliable).\\
\indent Prior to compositing, we converted each GHD series to days per year anomaly, relative to their local mean for 1600--1900. Despite the broad geographic range and climates gradients covered by these sites, there was good cross site correlation in the harvest dates (Supplemental Table 3; Supplemental Figure 3). Average harvest dates for all regional series, as well as GHD-Core and GHD-All (a composite average of all 27 sites), were anomalously early during the recent 1981--2007 interval relative to the baseline averaging period of 1600--1900, ranging from on average $-2$ days (Cha1) to over $-23$ day (SWi) early (Supplemental Table 4). There were also small differences across time in the inter-annual standard deviation in harvest dates (Supplemental Table 5), with most sites showing slightly reduced variability during the twentieth century compared to 1600--1900.

\subsection{Climate Data and Reconstructions.}
\noindent Instrumental temperature and precipitation data for the twentieth century (1901--2012) were taken from version 3.21 of the CRU climate grids\cite{Harris2014}. These data were monthly gridded fields, interpolated over land from individual station observations to a spatially uniform half degree grid. We also used a drought index, an updated version of the Palmer Drought Severity Index (PDSI\cite{Palmer:1965}) derived from the CRU data\cite{Schrier2013}. PDSI is a locally standardized indicator of soil moisture, calculated from inputs of precipitation and evapotranspiration. PDSI integrates precipitation over multiple months and seasons (about 12 months), and so it incorporates longer term changes in moisture balance beyond the immediate months or season.\\
\indent To extend our analysis further back in time, we also used three largely independent proxy based reconstructions of temperature\cite{Luterbacher2004}, precipitation\cite{Pauling2006}, and PDSI\cite{CookOWDA2015}. The temperature and precipitation products are 3-month seasonal reconstructions (DJF, MAM, JJA, SON) using primarily historical documentary evidence over the last 500 years. The temperature reconstruction covers the period 1500--2002; the precipitation reconstruction covers 1500--2000. The PDSI reconstruction is summer season only (JJA) and is based entirely upon tree ring chronologies distributed across Europe. It covers the entire Common Era, up through 2012. Prior to comparisons with the GHD data, we anomalized all three reconstruction products to a zero mean over 1600--1900, the same baseline period used in the GHD data.

\subsection{Wine Quality Data \& Analyses.}
% BIC 29-Jun-2015: Climate ref good!
\noindent We extracted wine quality data from Broadbent (2002)\cite{Broadbent2002}, which was ideal for our analyses in that it represented quality assessed by 1) a single observer who 2) attempted to correct for age since vintage in his ratings. Ratings were scaled from 0 to 5, with 0 indicating a `poor' vintage and 5 indicating an `outstanding' vintage. We extracted data for the 1900-2001 vintages in Bordeaux and Burgundy (2001 being the last year of data in the book). We selected these two regions for analysis because they are two of France's major wine-growing regions, coinciding with two major time-series of GHD included in GHD-Core, and represented the most serially complete time series (99\% for red Bordeaux, 98\% for white Bordeaux, 88\% for Red Burgundy and 59\% for white Burgundy, with almost all the missing data occurring before 1950). We fit ordered logit models to wine quality and CRU 3.21 climate data for each region by wine color (red or white), using the package \verb|ordinal| in \verb|R| 3.1.2\cite{Rcore2014}.

% We could note that any biases we would have with age-since-vintage affecting our results should go the opposite way: usually older vintages get better ratings because they have aged while we see the reverse. As I said above Broadbent attempts to correct for age since vintage already.
% We could also cite papers showing ratings are highly consistent across raters, but that would burn a reference so we could wait on that until/if needed.
\end{methods}


%% Put the bibliography here, most people will use BiBTeX in
%% which case the environment below should be replaced with
%% the \bibliography{} command.

% \begin{thebibliography}{1}
% \bibitem{dummy} Articles are restricted to 50 references, Letters
% to 30.
% \bibitem{dummyb} No compound references -- only one source per
% reference.
% \end{thebibliography}

\bibliographystyle{naturemag}
\bibliography{/Users/bcook/Dropbox/LATEX/mylib.bib}   % name your BibTeX data base

%% Here is the endmatter stuff: Supplementary Info, etc.
%% Use \item's to separate, default label is "Acknowledgements"
\pagebreak 
\begin{addendum}
 \item [Supplementary Information] is linked to the online version of the paper at www.nature.com/nature.
 \item[Competing Interests] The authors declare that they have no competing financial interests.
\item [Author Contributions:] BIC and EMW conceived of the paper and contributed equally to the writing. BIC conducted the climate analyses and processing of the harvest data, with contributions from EMW. EMW performed the wine quality analysis.
 \item[Correspondence] Correspondence and requests for materials
should be addressed to B.I.C.~(email: benjamin.i.cook@nasa.gov).
 \item[Acknowledgements] I. Garc\'ia de Cort\'azar-Atuari for help with Daux data, H. Eyster, S. Gee \& J. Samaha for extracting wine quality data.
\end{addendum}


\end{document}
