%% Template for a preprint Letter or Article for submission
%% to the journal Nature.
%% Written by Peter Czoschke, 26 February 2004
%%

\documentclass{nature}
\usepackage{url,hyperref}

%% make sure you have the nature.cls and naturemag.bst files where
%% LaTeX can find them

\bibliographystyle{naturemag}

%\title{Climate change decouples drought from early wine grape harvests across France}
\title{Climate change decouples drought from early wine grape harvests in Western Europe}

%% Notice placement of commas and superscripts and use of &
%% in the author list

\author{Benjamin I Cook$^{1,2}$ \& Elizabeth M Wolkovich$^{3,4}$}


\begin{document}

\maketitle

\begin{affiliations}
 \item NASA Goddard Institute for Space Studies, New York City, NY, USA
 \item Ocean and Climate Physics, Lamont-Doherty Earth Observatory, Palisades, NY, USA
 \item Arnold Arboretum, Boston, MA, USA
 \item Organismic and Evolutionary Biology, Harvard University, Cambridge, MA, USA
\end{affiliations}

\begin{abstract}
Climate change has advanced the timing of wine grape maturation and harvest dates (GHD) worldwide, with many regions witnessing advances of several weeks to over a month\cite{Duchene:2005bd,Seguin2005,webb2011}. Understanding the climatic drivers and longer-term context of recent GHD trends requires long-term records of both harvest dates and the associated climatic drivers, temperature and drought. Here, we combine long-term GHD records from across France\cite{Daux2012} with independent reconstructions of temperature\cite{Luterbacher2004} and drought\cite{CookOWDA2015,Pauling2006} to investigate the climatic controls of early harvest dates from from 1600--2007. We demonstrate that high temperatures and drought during the late spring and early summer (May-June-July) are the primary drivers of early harvests, but that in recent decades (1981--2007) drought has become largely decoupled from harvest timing. This decoupling is likely due to recent, anthropogenically forced warming trends providing large amounts of heat for fruit maturation; temperatures that, before significant anthropogenic warming, would have required a regional drought. Our results indicate that anthropogenic climate change may have fundamentally altered the drivers of early winegrape harvests across France, with possible ramification for both viticulture management and wine quality.
\end{abstract}

%Winegrapes (\emph{Vitis vinifera} ssp. \emph{vinifera}) are the world's most valuable horticultural crop (\href{http://faostat.fao.org}{http://faostat.fao.org}) and have shown dramatic responses to climate change over recent decades. 
\noindent Wine grapes (\emph{Vitis vinifera} ssp. \emph{vinifera}) are the world's most valuable horticultural crop, and there is emerging evidence that warming trends have advanced harvest dates in recent decades. For example, both Europe\cite{Jones:2000br,schultzjones,tomasi2011,odo2012} and Australia\cite{webb2012} have witnessed shifts towards earlier harvests, with some regions advancing several weeks to over a month since the 1970s\cite{Duchene:2005bd,Seguin2005,webb2011}. Concomitant with these phenological changes have also been shifts in wine quality ratings\cite{jones2005} and other metrics related to wine quality\cite{mori2007}, including sugar and acid levels at grape maturity\cite{Jones:2000br}.\\
\indent Both temperature and precipitation influence wine grape phenology, although wind, light, and other abiotic factors filtered through the local environment (i.e., \emph{terroir}) may also play a role\cite{Gladstones2011}. Warmer temperatures generally speed up grape vine phenological development from flowering to fruit maturation and harvest, while increased precipitation tends to delay these events\cite{jones2013}. Earliest harvests are thus generally found in years where the growing season is characterized by anomalously high temperatures and drought\cite{Jones:2000br}. Within these broad outlines, however, both temperature and precipitation can have highly variable effects, depending on their timing and magnitude. For example, high temperatures can damage leaf and grape tissues\cite{greer2010,Gladstones2011}, while heavy rain can burst grape clusters or promote rot\cite{jones2013}. An ideal harvest is thus favored by a long, warm summer characterized by early-season rains, combined with a late season drought. This ensures the vines and grape have sufficient heat and moisture to grow and mature early on, with dry conditions later in the year shifting them away from vegetative growth and towards greater investment in fruit production mid-season\cite{chaves2010,jones2013,baciocco2014}. Overall, both precipitation \cite{vanlee2009} and temperature \cite{baciocco2014} appear to control wine quality and the timing of harvest \cite{odo2012,webb2012}, though temperature is sometimes suggested to be more critical to wine grape phenology\cite{coombe1987} and many phenological analyses have focused solely on temperature\cite{jones2005}.\\
\indent Most previous investigations into climate change impacts on wine phenology have generally focused on recent, relatively short timescales (e.g., 30--40 years\cite{Duchene:2005bd,tomasi2011,webb2012}), and have not explicitly considered the stationarity (or lacker thereof) of wine phenology and climate relationships. To address this gap, here we analyze over 400 years of wine grape phenology records from Western Europe\cite{Daux2012}, comparing against both instrumental climate data for the 20\textsuperscript{th} century, and proxy based reconstructions of temperature\cite{Luterbacher2004}, precipitation\cite{Pauling2006}, and the Palmer Drought Severity Index\cite{CookOWDA2015} (PDSI) for the last several centuries.\\
\indent From the GHD database of Daux et al 2012, we constructed a multi-site GHD index (hereafter, GHD-Core) by averaging harvest date anomalies from 7 individual sites across France and Switzerland (see Methods for more details). This series (Figure 1) shows pronounced variability on both inter-annual and centennial scales over the last 400 years. The earliest date in the record is 1816, the so called `Year without a Summer' following the eruption of Mount Tambora in Indonesia\cite{Oppenheimer2003}, when cold conditions dominated continental Europe and harvest dates were +24.8 days late . The earliest date is 2003 (-33.4 days early) during one of the worst summer droughts and heat waves in recent history\cite{Rebetz2006}. During the first half of the twentieth century (1901--1950), harvest dates were modestly early (-4.49 days), while during the middle of the century (1951--1980) the were about average (-0.47 days). In recent decades (1981--2007), however, there is a strong trend towards earlier harvest dates, with average GHDs during this interval of -10.6 days. This early shift is in step with acceleration of European warming trends (REF XXXX), is significantly earlier (Student's t-test, $p\le0.0001$) compared to the previous interval (1600--1980), and the average even exceeds the baseline averaging period (1600--1900) standard deviation of -7.8 days.\\
\indent Coinciding with this accelerating trend towards earlier dates is also an apparent change in the strength of the climate relationships (for individual regional GHD series, see Supplemental Figures 4--8). Our core GHD index is most closely correlated (Spearman's rank) with temperature during the May-June-July interval, with some apparent weakening in the relationship in recent years (Figures 2, 3). As expected, GHD is universally negatively correlated with MJJ temperature, remaining consistently strong through 1980 (Figure 2, top row; Figure 3, left column). From 1981-2007, this correlation weakens, but remains significant and negative. Moreover, despite the weakening correlation strength and R2, the slope of the relationship (best-fit linear regression), remains about the same, between --6 and --7 days per degree of warming. This suggests that the sensitivity of GHD in the core index is relatively stationary.\\
\indent This is in sharp contrast to the apparent changing relationship between MJJ precipitation and soil moisture (as reflected in the PDSI). Both PDSI and precipitation are positively correlated with GHDs from 1901--1980, indicating dry conditions during MJJ drive earlier harvests. This may be due to direct drought impacts on fruit maturation (ref XXXX), but is more likely due to indirect influences via land-atmosphere feedbacks, namely the tendency for dry conditions to lead to warmer temperatures and vice versa  over this region (ref XXXX). Since 1981, however, this relationship has become insignificant and largely disappeared. Positive correlations over France pretty much universally disappear (Figure 2), and the regressions become insignificant and the slope of the regression lines pretty much flat (Figure 3). A similar result is seen when looking at June-July-August climate (JJA) (Supplemental Figure 4).\\
\indent 

\indent We composite all years with early GHD, defined as 1 standard deviation or earlier (-7.8 days, calculated from the 1600--1900 base period). From 1600--1980, this gives us 68 years and 18 years from 1981--2007. Average PDSI for these early years prior to 1980 is -1.04, indicative of mild drought conditions; from 1981 onward, it's actually slightly positive (+0.85). Median PDSI before and after is -0.87 and 0.96. Pooled PDSI from before and after are significantly different ($p\le0.001$) based on Student's t-test and Wilcoxon Ranksum test, indicating strongly that extreme early years needed drought conditions in the past at least based on PDSI. For JJA precipitation, there are only 11 years after 1981. Before, mean precip anomaly was --12\%; median -10\%. After 1980, mean was -1.3\% and median was -1.5\%. Differences in precipitation are only marginally significant: Student's t-test one tail (before drier than recent) $p=0.08$, Wilcoxon ranksum not significant. before 1980 is significantly drier than zero, after is not. Monte-Carol shows sensitivity to sampling.\\
\indent 

\indent To test for sampling uncertainty, we conducted 10,000 Monte-Carlo simulations in which we resampled (with replacement) PDSI, precipitation, and temperature from the early years before and after the 1980 break and recalculated our Student's t and Wilcoxon rank sum tests. Both before and after, we resampled the number of years available after 1981: 18 years (PDSI), XX years (precipitation), and XX years (temperature). For PDSI, 95\% of the simulations were still significantly different Student's t-test; for Wilcoxon it was 89\%.

%Then the body of the main text appears after the intro paragraph.
%Figure environments can be left in place in the document.
%\verb|\includegraphics| commands are ignored since Nature wants
%the figures sent as separate files and the captions are
%automatically moved to the end of the document (they are printed
%out with the \verb|\end{document}| command. However, tables must
%be manually moved to the end of the document, after the addendum.
%\noindent Wine grapes (\emph{Vitis vinifera} ssp. \emph{vinifera}) are the world's most valuable horticultural crop, and there is emerging evidence that harvest dates have been advancing in recent decades in association with warming trends. Across Europe\cite{Jones:2000br,schultzjones,tomasi2011,odo2012} and Australia\cite{webb2012}, there has been a shift towards earlier harvest dates with some regions reporting an advance of several weeks to over a month since the 1970s\cite{Duchene:2005bd,Seguin2005} and others an advance of over two weeks per decade\cite{webb2011}. Concomitant with these changes have been shifts in wine quality ratings \cite{jones2005} and other metrics connected to wine quality\cite{mori2007}, including sugar and acid levels at grape maturity\cite{Jones:2000br}. The timing of wine grape harvest is linked closely with the phenological event of veraison, the ripening and coloration of the fruit, with earlier harvests generally associated with warm(REF XXXX) and dry(ref XXXX) conditions during the growing season.\\
%\indent Temperature and precipitation are expected to combine to influence the timing of these events, though wind, light and how all these abiotic forces are experienced by vines through the full expression of the local environment (or \emph{terroir}) also play a role \cite{Gladstones2011}. Warmer temperatures generally speed phenological events from flowering to harvest and greater precipitation generally delays events \cite{jones2013}, thus leading to the earliest harvests generally in years where growing seasons were characterized by higher temperatures and drought \cite{Jones:2000br}. Within these broad outlines both temperature and precipitation can have highly variable effects, depending on their timing and how extreme they are. For example, high temperatures can damage leaf and grape tissues \cite{greer2010,Gladstones2011} and heavy rain can burst grape clusters or promote rot \cite{jones2013}. An ideal harvest is expected from a long, warm summer---giving vines and their grapes sufficient thermal time to grow and mature---with sufficient early-season rains followed by a later season drought, giving vines resources to grow in the early season, but shifting them away from vegetative growth and towards greater investment in fruit production mid-season \cite{chaves2010,jones2013,baciocco2014}. Overall, both precipitation \cite{vanlee2009} and temperature \cite{baciocco2014} appear to control wine quality and the timing of harvest \cite{odo2012,webb2012}, though temperature is sometimes suggested to be more critical to winegrape phenology \cite{coombe1987} and many phenological analyses have focused solely on temperature \cite[e.g.,][]{jones2005}.\\

\begin{figure}
\caption{Left panel: time series of Grape Harvest Date (GHD) anomalies from GHD-Core, composited from the Als, Bor, Bur, Cha1, LLV, SRv, and SWi regional time series in DAUX. All anomalies are in units of day of year, relative to the average date calculated from 1600--1900. Right panel: normalized histograms of GHD anomalies from the core index for 1600--1980 and 1981--2007.}
\end{figure}

\begin{figure}
\caption{Point-by-point correlations (Spearman's rank) between GHD-Core and May-June-July temperature, precipitation, and Palmer Drought Severity Index (PDSI) for three periods: 1901--1950, 1951--1980, and 1981--2007. All climate data are from the CRU 3.21 climate grids, described in the Methods section. Prior to calculating the correlations, we linearly detrended both the climate data and GHD-Core.}
\end{figure}

\begin{figure}
\caption{Linear regressions between regional average climate variables (2\textsuperscript{o}W--8\textsuperscript{o}E, 43\textsuperscript{o}N--51\textsuperscript{o}N)}
\end{figure}


%In addition, a cover letter needs to be written with the
%following:
%\begin{enumerate}
% \item A 100 word or less summary indicating on scientific grounds
%why the paper should be considered for a wide-ranging journal like
%\textsl{Nature} instead of a more narrowly focussed journal.
% \item A 100 word or less summary aimed at a non-scientific audience,
%written at the level of a national newspaper.  It may be used for
%\textsl{Nature}'s press release or other general publicity.
% \item The cover letter should state clearly what is included as the
%submission, including number of figures, supporting manuscripts
%and any Supplementary Information (specifying number of items and
%format).
 %\item The cover letter should also state the number of
%words of text in the paper; the number of figures and parts of
%figures (for example, 4 figures, comprising 16 separate panels in
%total); a rough estimate of the desired final size of figures in
%terms of number of pages; and a full current postal address,
%telephone and fax numbers, and current e-mail address.
%\end{enumerate}

%See \textsl{Nature}'s website
%(\texttt{http://www.nature.com/nature/submit/gta/index.html}) for
%complete submission guidelines.

\begin{methods}
\subsection{Grape Harvest Data.}
\noindent Grape harvest dates are taken from the Daux et al 2012 database (hereafter, DAUX) of wine harvest dates from western Europe\cite{Daux2012}. DAUX is composed of 27 regional composite GHD time series, mostly from France, but also including series from Switzerland, Spain, Luxembourg, and Germany (Supplemental Figure 1). These data are ideal for climate change research problems because, at least for France, management changes over time are relatively minor and irrigation of vineyards is quite rare. Indeed, these data have been used previously for development of proxy-based temperature reconstructions\cite{Daux2012}. Rather than focus our analysis on the individual regional time series, we created a composite average core index from several regional series (GHD-Core). Focus on a composite series like GHD-Core has several advantages. First, every regional GHD series has missing values; by averaging multiple sites into a single composite, we ensure a serially complete time series. Second, because viticulture management varies across grape varietals and regions, use of a composite average series should minimize management effects and emphasize larger scale signals related to climate variability and change.\\
\indent From the 27 sites available, we chose 7 sites (Supplemental Table 1) to construct the GHD-Core index that each provided reasonably complete data coverage after 1981. These sites are Alsace (Als), Bordeaux (Bor), Burgundy (Bur), Champagne 1 (Cha1), the Lower Loire Valley (LLV), the Southern Rhone Valley (SRv), and Switzerland at Leman Lake (Swi). After 1600, most years have at least 3-4 of these regional series represented; sample depth declines sharply prior to this date (Supplemental Figure 2). All analyses are thus restricted to the period from 1600--2007. Prior to compositing, we converted each series to anomalous days per year, relative to the mean for 1600-1900. Despite the broad geographic range and climates gradients covered by these sites, there is good cross site correlation in the harvest dates (Supplemental Table 2; Supplemental Figure 3). Average harvest dates for all regional series, as well as GHD-Core and GHD-All (a composite average of all 27 sites), are anomalously early during the recent 1981--2007 interval relative to 1600--1900, ranging from on average 2 days (Cha1) to over 3 weeks (SWi) (Supplemental Table 3). There are also small differences across time in the inter-annual standard deviation in harvest dates (Supplemental Table 4), with most sites showing slightly reduced variability during the twentieth century compared to 1600--1900.
%Each region level GHD series is anomalized relative to its' mean from 1600--1900 before averaging together to form GHD-Core (all series are weighted equally). By using a multi-site composite series, we can fill in site level missing years so that we are serially complete. Also we reduce the influence of local management changes by emphasizing the shared climate signal across sites.
%Estimates of drought variability over the historical period and last millennium used the latest version of the North American Drought Atlas (NADA)\cite{Cook:etal2007a}, a tree-ring based reconstruction of summer season (June-July-August) PDSI. All statistics were based on regional PDSI averages over the Central Plains (105\textsuperscript{o}W--92\textsuperscript{o}W, 32\textsuperscript{o}N--46\textsuperscript{o}N) and the Southwest (125\textsuperscript{o}W--105\textsuperscript{o}W, 32\textsuperscript{o}N--41\textsuperscript{o}N). We restricted our analysis to 1000-2005 CE; prior to 1000 CE, the quality of the reconstruction in these regions declines.

\subsection{Climate Data and Reconstructions.}
\noindent Instrumental temperature and precipitation data for the twentieth century (1901--2012) are taken directly from version 3.21 of the CRU climate grids\cite{Harris2014}. These data are monthly gridded fields, interpolated over land from individual station observations to a spatially uniform half degree grid. We also use a drought index, the Palmer Drought Severity Index (PDSI\cite{Palmer:1965}), derived from the CRU data\cite{Schrier2013}. PDSI is a locally standardized indicator of soil moisture, calculated from inputs of precipitation and evapotranspiration. PDSI integrates precipitin over multiple months and seasons (about 12 months), so incorporates longer term changes in 
%The 21\textsuperscript{st}-century drought projections used output from 17 GCMs in phase 5 of the Coupled Model Intercomparison Project\cite{Taylor2012} (Supplemental Table 1). All models represent one or more continuous ensemble members from the historical (1850--2005 CE) and RCP 8.5 (2006--2099 CE) emission scenarios. We used the same methodology as Ref. 2 to calculate model PDSI for the full interval (1850--2099 CE), using the Penman-Monteith formulation of potential evapotranspiration. The baseline period for calibrating and standardizing the model PDSI anomalies was 1931--1990 CE, the same baseline period as the NADA PDSI. Negative model PDSI values therefore indicate drier conditions than the 1931-1990 period.\\
\end{methods}


%% Put the bibliography here, most people will use BiBTeX in
%% which case the environment below should be replaced with
%% the \bibliography{} command.

% \begin{thebibliography}{1}
% \bibitem{dummy} Articles are restricted to 50 references, Letters
% to 30.
% \bibitem{dummyb} No compound references -- only one source per
% reference.
% \end{thebibliography}

\bibliographystyle{naturemag}
\bibliography{/Users/bcook/Dropbox/LATEX/mylib.bib}   % name your BibTeX data base

%% Here is the endmatter stuff: Supplementary Info, etc.
%% Use \item's to separate, default label is "Acknowledgements"

\begin{addendum}
 \item [Supplementary Information] is linked to the online version of the paper at www.nature.com/nature.
 \item[Competing Interests] The authors declare that they have no
competing financial interests.
%\item [Author Contributions:] BIC wrote the paper, calculated PDSI and standardized soil moisture from the models, and conducted the bulk of the analyses. TRA calculated the risk percentages and contributed to idea development, writing, and interpretation of results. JES contributed to idea development, writing, and interpretation of results.
 \item[Correspondence] Correspondence and requests for materials
should be addressed to B.I.C.~(email: benjamin.i.cook@nasa.gov).
\end{addendum}


\end{document}
