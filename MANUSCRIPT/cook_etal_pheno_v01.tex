%% Template for a preprint Letter or Article for submission
%% to the journal Nature.
%% Written by Peter Czoschke, 26 February 2004
%%

\documentclass[final]{nature}
\usepackage{url,hyperref}
\usepackage{graphicx}


%% make sure you have the nature.cls and naturemag.bst files where
%% LaTeX can find them

\bibliographystyle{naturemag}

%\title{Climate change decouples drought from early wine grape harvests across France}
% EMW says: I don't know that 'Western Europe' (Spain, Italy) will like this -- I think we should stick with France even though it excludes Switzerland ... or just skip location?!
\title{Climate change decouples drought from early wine grape harvests}

%% Notice placement of commas and superscripts and use of &
%% in the author list

\author{Benjamin I Cook$^{1,2}$ \& Elizabeth M Wolkovich$^{3,4}$}


\begin{document}

\maketitle

\begin{affiliations}
 \item NASA Goddard Institute for Space Studies, New York City, NY, USA
 \item Ocean and Climate Physics, Lamont-Doherty Earth Observatory, Palisades, NY, USA
 \item Arnold Arboretum, Boston, MA, USA
 \item Organismic and Evolutionary Biology, Harvard University, Cambridge, MA, USA
\end{affiliations}

\begin{abstract}
Climate change has advanced the timing of winegrape maturation and harvest worldwide, with many regions witnessing grape harvest dates (GHD) today that are several weeks to over a month early \cite{Duchene:2005bd,Seguin2005,webb2011}. Understanding ties between climate change and these earlier harvests requires teasing out the often intertwined drivers of fruit maturation: temperature and drought in a longer-term context that includes sufficient data both before and after the start of anthropogenic shifts in the climate system. Here, we combine historical GHD records from across Western Europe\cite{Daux2012} with independent reconstructions of temperature\cite{Luterbacher2004} and drought\cite{CookOWDA2015,Pauling2006} to investigate the climatic controls of early harvest dates from 1600--2007. We demonstrate that high temperatures and drought during the late spring and early summer (May-June-July) are the primary drivers of early harvests, but that in recent decades (1981--2007) drought has become largely decoupled from harvest timing. This decoupling is likely related to a decoupling of the climate itself: before significant anthropogenic warming high summer temperatures---that could speed fruit maturation---generally required drought conditions, on contrast recent higher temperatures often occur without drought conditions. Our results indicate that anthropogenic climate change may have fundamentally altered the climatic drivers of early winegrape harvests across Western Europe, with possible ramifications for both viticulture management and wine quality. % booyah
\end{abstract}

%Winegrapes (\emph{Vitis vinifera} ssp. \emph{vinifera}) are the world's most valuable horticultural crop (\href{http://faostat.fao.org}{http://faostat.fao.org}) and have shown dramatic responses to climate change over recent decades. 
\noindent Winegrapes (\emph{Vitis vinifera} ssp. \emph{vinifera}) are the world's most valuable horticultural crop, and there is increasing evidence that warming trends have advanced winegrape harvest dates in recent decades\cite{Jones:2000br,schultzjones,tomasi2011,odo2012,webb2012}. In many regions, harvests are occurring several weeks to over a month earlier compared to the 1970s\cite{Duchene:2005bd,Seguin2005,webb2011}. Concomitant with these phenological changes are shifts in wine quality ratings\cite{jones2005} and other metrics related to wine quality\cite{Jones:2000br,mori2007}. Both temperature and precipitation influence winegrape phenology, although wind, light, and other abiotic factors filtered through the local environment (i.e., \emph{terroir}) may also play a role\cite{Gladstones2011}. Warmer temperatures generally accelerate grape vine phenological development from flowering to fruit maturation and harvest, while increased precipitation tends to delay these events\cite{jones2013}. The earliest harvests thus generally occur in years where the growing season is characterized by higher than average temperatures and drought\cite{Jones:2000br}.\\
% EMW says: If needed in the future I think we could trim some of the above and below to make the topic sentences all very on point. Let me know what you think and I can work on it. For now I just adjusted the below sentence to not be so focused on extreme events, since we don't look at those as much.
\indent Within these broad climatic constraints, however, are the nuances of an early and ideal year versus a late, poor vintage. Extreme temperature and precipitation events may have additional effects, depending on their timing and magnitude. For example, high temperatures can damage leaf and grape tissues\cite{greer2010,Gladstones2011}, while heavy rains can burst grape clusters or promote rot\cite{jones2013}. An ideal harvest is thus typically favored by warm summers with above average early-season rains combined with a late season drought. This ensures the vines and grapes have sufficient heat and moisture to grow and mature early on, with dry conditions later in the year shifting them away from vegetative growth and towards greater investment in fruit production mid-season\cite{chaves2010,jones2013,baciocco2014}. Overall, both precipitation\cite{vanlee2009} and temperature\cite{baciocco2014} appear to control wine quality and the timing of harvest\cite{odo2012,webb2012}, though temperature is suggested to be most critical to winegrape phenology\cite{coombe1987,jones2005}.\\
% Should we add one sentence above that temperature is so critical to winegrape phenology that it has been used to back forecast climate? 
% Need a better topic sentence below eventually. We need to transition more to topic of our paper before getting to timescale of paper versus others. I took a stab at it but I know it needs work.
\indent Given the importance of climate on winegrape harvests it is perhaps not surprising that grape harvest dates (GHD) have experienced some of the greatest advances of crops with climate change. Many investigations of climate change impacts on wine phenology have documented dramatic shifts in GHD over the several decades, and tied these to shifts in climate. Such work, however, has also generally focused on recent, relatively short timescales (e.g., 30--40 years\cite{Duchene:2005bd,tomasi2011,webb2012}) and thus has not explicitly considered the stationarity (or lack thereof) of the relationship between winegrape phenology and climate. To address this gap, we analyze over 400 years (1600--2007) of winegrape phenology records from Western Europe\cite{Daux2012}, comparing against both instrumental climate data for the 20\textsuperscript{th} century\cite{Harris2014} and proxy based reconstructions over the last 400 years of temperature\cite{Luterbacher2004}, precipitation\cite{Pauling2006}, and an index of soil moisture, the Palmer Drought Severity Index\cite{CookOWDA2015} (PDSI).\\
\indent From the GHD database of Daux et al 2012\cite{Daux2012}, we constructed a multi-site GHD index (hereafter, GHD-Core) by averaging harvest date anomalies from 7 individual sites across France and Switzerland (see Methods for more details). This series (Figure 1, left panel) shows pronounced variability on both inter-annual and centennial scales over the last 400 years. The latest GHD anomaly in the record is 1816, the so called `Year without a Summer' following the eruption of Mount Tambora in Indonesia\cite{Oppenheimer2003}. The eruption caused pronounced cooling over continental Europe with harvest dates in GHD-Core delayed over three weeks (+24.8 days). The earliest date in the record is 2003 (-33.4 days early), coinciding with one of the worst summer droughts and heat waves in recent history\cite{Rebetz2006}. During the first half of the twentieth century (1901--1950), harvest dates were modestly early (-4.49 days), while during the middle of the century (1951--1980) they were about average (-0.47 days) (Supplemental Table 4). In more recent decades (1981--2007), however, that has been a strong shift towards earlier harvest dates, on average -10.6 days earlier (Figure 1, right panel). This shift is significant compared to the previous interval (1600--1980; Student's t-test, $p\le0.0001$), even exceeding one standard deviation (7.8 days) of GHD variability during the baseline averaging period (1600--1900).\\
\indent Coinciding with this recent shift towards earlier harvest dates is also an apparent change in the strength of the relationships between GHD-Core and various climate variables in the CRU climate grids (for individual regional GHD series, see Supplemental Figures 4--10). Over the first half of the twentieth century (1901--1950), GHD-Core correlates negatively (Spearman's rank) with May-June-July (MJJ) temperatures across Western Europe (Figure 2, top row), indicating a strong tendency for earlier harvests during warmer conditions in late spring and early summer. Correlations are positive, though weaker, with MJJ precipitation (Figure 2, middle row) and PDSI (Figure 2, bottom row), indicating earlier harvests during drought.\\
\indent These relationships persist into the middle of the century (1951--1980) but, in the case of precipitation and PDSI, break down in recent decades (1981-2007). Both before and after 1980 (Figure 3), regional average (dashed box in Figure 2; 2\textsuperscript{o}W--8\textsuperscript{o}E, 43\textsuperscript{o}N--51\textsuperscript{o}N) MJJ temperatures are the single best predictor of GHD-Core anomalies, explaining nearly 70\% of the variance for 1901--1980 and only weakening slightly in the more recent period ($R^2=0.57$). Notably, the slope of the relationship remains about the same, between --6 and --7 days advancement in GHD per degree of warming, suggesting that the temperature sensitivity of harvest dates is relatively stationary. This is in sharp contrast to the apparent changing relationship between GHD-Core and MJJ precipitation and PDSI. Both PDSI and precipitation are positively correlated with GHDs from 1901--1980, indicating below average precipitation and drought conditions during MJJ will lead to earlier harvests. This may be due to direct drought impacts on fruit maturation by increasing abscisic acid production\cite{webb2012}, but may also arise indirectly via feedbacks between soil moisture and air temperature. When soils are dry, surface energy partitioning favors sensible over latent (evapotranspiration) heating, increasing soil and air temperatures. Western Europe is a region where this feedback is thought to be especially strong\cite{Seneviratne2006}. Indeed, the relationship between temperature and moisture (precipitation and PDSI) over the GHD-Core region during MJJ is negative (Supplemental Figure 11, top row). Since 1981, however, the relationship between GHD and these moisture variables has become insignificant: positive correlations over the GHD-Core region largely disappear (Figure 2), and the slope of the regression lines become indistinguishable from zero (Figure 3).\\
\indent To further explore this apparent non-stationarity, we calculated composite average climate anomaly maps from the climate reconstructions for years when GHD occurred -7.8 days early or earlier (one standard deviation). Instead of MJJ, we used June-July-August (JJA), the closest match available given the seasonal resolution of the reconstruction products. In the instrumental data, the relationship between GHD-Core and JJA temperature and precipitation weakens, while PDSI improves slightly (all are still significant; Supplemental Figure 12). As with MJJ, there is a sharp breakdown in the GHD-moisture regressions after 1980. Additionally, the temperature-moisture coupling relationship is still significant, and actually improves for JJA (Supplemental Figure 13).\\
\indent This yielded a composite of 68 early GHD years from 1600--1980; from 1981--2007 the composite ranged from 11--18 years, depending on the end date of the different climate reconstructions (Figure 4). Early harvests are associated with warmer than average conditions in both intervals, increasing in intensity in the more recent period, consistent with large-scale warming trends over Europe. In sharp contrast, the typical dry anomalies associated with early harvests from 1600--1980 effectively disappear in the more recent interval. Before 1980, regional average PDSI values during early harvests were -1.04 and precipitation was -12\% below normal, indicative drought conditions. Since 1980, however, mean precipitation during early harvests was only slightly below normal (-1.3\%) and PDSI was wetter than average (+0.86). For PDSI, these differences are highly significant (One Sided Student's t-test, $p\le0.001$), while only marginally significant for precipitation (One Sided Student's t-test, $p=0.08$). However, a one sample Student's t-test comparing the precipitation anomalies against a mean of zero found that only the precipitation anomalies in pre-1980 period are significantly below normal. These results were confirmed by a Monte-Carlo analysis to test for sampling uncertainties in the composite averaging (Supplementary Material, Supplemental Figure 14). Combined with the earlier instrumental climate analysis, these results further support our conclusion that drought has been decoupled in recent decades as a climate driver of early harvest dates.\\
\indent Finally, as harvest dates are generally coupled to wine quality in France we tested for shifting relationships between wine quality in Bordeaux and Burgundy (CITE Broadbent, citation info in methods below) and climate from 1900-1980 and 1981-2001. Across these time periods GHD is a strong predictor of wine quality with earlier harvest dates leading to greater odds of a higher quality wine (see Table SX); the magnitude of this relationship is generally consistent before and after 1980 and across regions. Results considering wine quality and climate in many ways recapitulated our findings with GHD.  Higher temperatures and lower amounts of precipitation yield higher odds of a high quality wine (Table SX) and these relationships were generally consistent across regions and before and after 1980 (Table SX), though the relationship between precipitation and quality disappears after 1980 for red wines in Burgundy (Table SX). Lower soil moisture also increased the odds of a higher quality vintage, however---in line with our findings above---this was seen only before 1980. After 1980, in both Bordeaux and Burgundy significant relationships between quality and PDSI disappeared and the magnitudes decreased greatly (halved for red wines in each region, see Table SX). Taken together our results on quality suggest a consistent strong positive relationship between quality, earlier harvest dates and higher temperatures but a dramatically weakened relationship to drought conditions concomitant with climate change.\\ % Yeah, this is longer than we need but somewhere to start. Also I have been trying to figure out how best to report the ordinal results and I am just not 100% sure beyond what I wrote above. For example, I think the -0.1 GHD coefficient basically means for every 10 day advance in harvest you likely increase 1 level in quality but, try as I may, I have yet to confirm this for myself. I am not sure it adds that much anyway. 

\emph{Caveats we may want to add:} Quality results are variable across regions: temperature is a stronger predictor of quality in Bordeaux compared to Burgundy (that is, quality of Bordeaux wines increases about about twice as quickly with higher temperatures compared to wines from Burgundy, except for the weird strengthening of the relationship for whites in Burgundy after 1980). Variable responses are due to probably to a bunch of things including (a) microclimatic variation, which occurs at as small as scale as the sub-vineyard and which our analyses brush over, (b) variation in the y axis of our climate variables --- we should be sure to hit this one home as higher temperature may be good now, but -- if we bring the reader back to the introduction -- at their extremes they lead to vintage failures.

%Then the body of the main text appears after the intro paragraph.
%Figure environments can be left in place in the document.
%\verb|\includegraphics| commands are ignored since Nature wants
%the figures sent as separate files and the captions are
%automatically moved to the end of the document (they are printed
%out with the \verb|\end{document}| command. However, tables must
%be manually moved to the end of the document, after the addendum.
%\noindent Wine grapes (\emph{Vitis vinifera} ssp. \emph{vinifera}) are the world's most valuable horticultural crop, and there is emerging evidence that harvest dates have been advancing in recent decades in association with warming trends. Across Europe\cite{Jones:2000br,schultzjones,tomasi2011,odo2012} and Australia\cite{webb2012}, there has been a shift towards earlier harvest dates with some regions reporting an advance of several weeks to over a month since the 1970s\cite{Duchene:2005bd,Seguin2005} and others an advance of over two weeks per decade\cite{webb2011}. Concomitant with these changes have been shifts in wine quality ratings \cite{jones2005} and other metrics connected to wine quality\cite{mori2007}, including sugar and acid levels at grape maturity\cite{Jones:2000br}. The timing of wine grape harvest is linked closely with the phenological event of veraison, the ripening and coloration of the fruit, with earlier harvests generally associated with warm(REF XXXX) and dry(ref XXXX) conditions during the growing season.\\
%\indent Temperature and precipitation are expected to combine to influence the timing of these events, though wind, light and how all these abiotic forces are experienced by vines through the full expression of the local environment (or \emph{terroir}) also play a role \cite{Gladstones2011}. Warmer temperatures generally speed phenological events from flowering to harvest and greater precipitation generally delays events \cite{jones2013}, thus leading to the earliest harvests generally in years where growing seasons were characterized by higher temperatures and drought \cite{Jones:2000br}. Within these broad outlines both temperature and precipitation can have highly variable effects, depending on their timing and how extreme they are. For example, high temperatures can damage leaf and grape tissues \cite{greer2010,Gladstones2011} and heavy rain can burst grape clusters or promote rot \cite{jones2013}. An ideal harvest is expected from a long, warm summer---giving vines and their grapes sufficient thermal time to grow and mature---with sufficient early-season rains followed by a later season drought, giving vines resources to grow in the early season, but shifting them away from vegetative growth and towards greater investment in fruit production mid-season \cite{chaves2010,jones2013,baciocco2014}. Overall, both precipitation \cite{vanlee2009} and temperature \cite{baciocco2014} appear to control wine quality and the timing of harvest \cite{odo2012,webb2012}, though temperature is sometimes suggested to be more critical to winegrape phenology \cite{coombe1987} and many phenological analyses have focused solely on temperature \cite[e.g.,][]{jones2005}.\\

\begin{figure}
\caption{Left panel: time series of Grape Harvest Date (GHD) anomalies from GHD-Core, composited from the Als, Bor, Bur, Cha1, LLV, SRv, and SWi regional GHD time series in DAUX. All anomalies are in units of day of year, relative to the average date calculated from 1600--1900. Right panel: normalized histograms of GHD anomalies (day of year) from GHD-Core for two periods: 1600--1980 and 1981--2007.}
\end{figure}

\begin{figure}
\caption{Point-by-point correlations (Spearman's rank) between GHD-Core and May-June-July temperature, precipitation, and Palmer Drought Severity Index (PDSI) for three periods: 1901--1950, 1951--1980, and 1981--2007. All climate data are from the CRU 3.21 climate grids, described in the Methods section. Prior to calculating the correlations, we linearly detrended both the climate data and GHD-Core.}
\end{figure}

\begin{figure}
\caption{Linear regressions between GHD-Core and May-June-July climate variables from CRU 3.21, averaged over the main GHD-Core region (2\textsuperscript{o}W--8\textsuperscript{o}E, 43\textsuperscript{o}N--51\textsuperscript{o}N). Top row: 1901--1980. Bottom row: 1981--2007.}
\end{figure}

\begin{figure}
\caption{Composite temperature, precipitation, and PDSI anomalies from the various climate reconstructions (see Methods), for years with early harvest dates ($\le-7.8$ days early). Numbers in the lower left corners indicate the number of years used to construct the composite.}
\end{figure}

%In addition, a cover letter needs to be written with the
%following:
%\begin{enumerate}
% \item A 100 word or less summary indicating on scientific grounds
%why the paper should be considered for a wide-ranging journal like
%\textsl{Nature} instead of a more narrowly focussed journal.
% \item A 100 word or less summary aimed at a non-scientific audience,
%written at the level of a national newspaper.  It may be used for
%\textsl{Nature}'s press release or other general publicity.
% \item The cover letter should state clearly what is included as the
%submission, including number of figures, supporting manuscripts
%and any Supplementary Information (specifying number of items and
%format).
 %\item The cover letter should also state the number of
%words of text in the paper; the number of figures and parts of
%figures (for example, 4 figures, comprising 16 separate panels in
%total); a rough estimate of the desired final size of figures in
%terms of number of pages; and a full current postal address,
%telephone and fax numbers, and current e-mail address.
%\end{enumerate}

%See \textsl{Nature}'s website
%(\texttt{http://www.nature.com/nature/submit/gta/index.html}) for
%complete submission guidelines.

\begin{methods}
\subsection{Grape Harvest Data.}
\noindent We analyzed GHD data in the database of regional winegrape harvest time series from Western Europe compiled by Daux et al 2012 (hereafter, DAUX\cite{Daux2012}). DAUX includes 27 regional composite time series of winegrape harvest dates compiled from local vineyard and winery records going back as far as 1354. Most of these series are from France, but also included are data from Switzerland, Spain, Luxembourg, and Germany (Supplemental Figure 1). These data are ideal for climate change research applications because management practices have changed very little over time and irrigation as a viticulture tool (which could complicate the interpretation of climate relationships) is largely absent, especially in France. Indeed, these data have been used previously to develop proxy-based temperature reconstructions for the region\cite{Daux2012}.\\
\indent Rather than focus our analysis on the individual regional GHD series, we created a composite average index from several regional series (GHD-Core). Using a multi-site composite series has two main advantages. First, every regional GHD series has missing values. By averaging multiple sites into a single composite index, we were able to ensure a serially complete time series back to 1600. Second, because viticulture management varies across winegrape varietals and regions, use of a composite average series should minimize the influence of local management effects and instead emphasize larger scale signals related to climate variability and change (the primary focus of our study).\\
\indent From the 27 regional GHD series available, we chose 7 sites (Supplemental Table 1) to construct GHD-Core: Alsace (Als), Bordeaux (Bor), Burgundy (Bur), Champagne 1 (Cha1), the Lower Loire Valley (LLV), the Southern Rhone Valley (SRv), and Switzerland at Leman Lake (Swi). All 7 regional series are over 80\% serially complete back to 1800 and all but one (Cha1) are over 60\% complete back to 1600 (Supplemental Table 2). Additionally, all 7 sites have good coverage of the most recent period (1981--2007) when we conclude that drought controls on harvest date have significantly weakened. After 1600, most years have at least 3-4 of these regional series represented; sample depth declines sharply prior to this date (Supplemental Figure 2). All analyses are thus restricted to the period from 1600--2007.\\
\indent Prior to compositing, we converted each series to days per year anomaly, relative to their local mean for 1600-1900. Despite the broad geographic range and climates gradients covered by these sites, there is good cross site correlation in the harvest dates (Supplemental Table 3; Supplemental Figure 3).  Average harvest dates for all regional series, as well as GHD-Core and GHD-All (a composite average of all 27 sites), are anomalously early during the recent 1981--2007 interval relative to the baseline averaging interval of 1600--1900, ranging from on average 2 days (Cha1) to over 3 weeks (SWi) early (Supplemental Table 4). There are also small differences across time in the inter-annual standard deviation in harvest dates (Supplemental Table 5), with most sites showing slightly reduced variability during the twentieth century compared to 1600--1900.

\subsection{Climate Data and Reconstructions.}
\noindent Instrumental temperature and precipitation data for the twentieth century (1901--2012) are taken directly from version 3.21 of the CRU climate grids\cite{Harris2014}. These data are monthly gridded fields, interpolated over land from individual station observations to a spatially uniform half degree grid. We also use a drought index, an updated version of the Palmer Drought Severity Index (PDSI\cite{Palmer:1965}) derived from the CRU data\cite{Schrier2013}. PDSI is a locally standardized indicator of soil moisture, calculated from inputs of precipitation and evapotranspiration. PDSI integrates precipitation over multiple months and seasons (about 12 months), and so it incorporates longer term changes in moisture balance beyond the immediate months or season.\\
\indent To extend our analysis further back in time, we also used three largely independent proxy based reconstructions of temperature\cite{Luterbacher2004}, precipitation\cite{Pauling2006}, and PDSI\cite{CookOWDA2015}. The temperature and precipitation products are 3-month seasonal reconstructions (DJF, MAM, JJA, SON) using primarily historical documentary evidence over the last 500 years. The temperature reconstruction covers the period 1500--2002; the precipitation reconstruction covers 1500--2000. The PDSI reconstruction is summer season only (JJA) and is based entirely upon tree ring chronologies distributed across Europe. It covers the entire Common Era, up through 2012. Prior to comparisons with the GHD data, we anomalized all three reconstruction products to a zero mean over 1600--1900, the same baseline period used in the GHD data.

\subsection{Wine Quality Data \& Analyses.}
We extracted wine quality data from (Broadbent, M. 'Vintage Wine: Fifty Years of Tasting Three Centuries of Wine' 2002, Harcourt Inc., New York: 560 pp.), which is ideal for our analyses in that it represents quality assessed by one observer, who attempts to correct for age since vintage in his ratings. Ratings are scaled from 0 to 5, with 0 indicating a `poor' vintage and 5 indicating an `outstanding' vintage. We extracted data for the vintages 1900-2001 in Bordeaux and Burgundy (2001 being the last year of data in the book). We selected these two regions for analysis because they are two of France's major wine-growing regions, coinciding with two major time-series of GHD included in GHD-Core, and represented the most serially complete time series (99\% for red Bordeaux, 98\% for white Bordeaux, 88\% for Red Burgundy and 59\% for white Burgundy with almost of the all the missing data occurring before 1950). We fit ordered logit models for each region x color using the package \verb|ordinal| in \verb|R| (add citation below).

R citation: A BibTeX entry for LaTeX users is

  @Manual{,
    title = {R: A Language and Environment for Statistical Computing},
    author = {{R Core Team}},
    organization = {R Foundation for Statistical Computing},
    address = {Vienna, Austria},
    year = {2014},
    url = {http://www.R-project.org/},
  }


% We could note that any biases we would have with age-since-vintage affecting our results should go the opposite way: usually older vintages get better ratings because they have aged while we see the reverse. As I said above Broadbent attempts to correct for age since vintage already.
% We could also cite papers showing ratings are highly consistent across raters, but that would burn a reference so we could wait on that until/if needed.
\end{methods}


%% Put the bibliography here, most people will use BiBTeX in
%% which case the environment below should be replaced with
%% the \bibliography{} command.

% \begin{thebibliography}{1}
% \bibitem{dummy} Articles are restricted to 50 references, Letters
% to 30.
% \bibitem{dummyb} No compound references -- only one source per
% reference.
% \end{thebibliography}

\bibliographystyle{naturemag}
\bibliography{/Users/bcook/Dropbox/LATEX/mylib.bib}   % name your BibTeX data base

%% Here is the endmatter stuff: Supplementary Info, etc.
%% Use \item's to separate, default label is "Acknowledgements"

\begin{addendum}
 \item [Supplementary Information] is linked to the online version of the paper at www.nature.com/nature.
 \item[Competing Interests] The authors declare that they have no
competing financial interests.
\item [Author Contributions:] BIC and EMW conceived of the paper and contributing equally to the writing. BIC conducted most of the analyses, while EMW conducted the wine quality analysis.
 \item[Correspondence] Correspondence and requests for materials
should be addressed to B.I.C.~(email: benjamin.i.cook@nasa.gov).
 \item[Acknowledgements] I Garcia de Cortazar for help with Daux data, H. Eyster, S. Gee \& J. Samaha for extracting wine quality data.
\end{addendum}


\end{document}
